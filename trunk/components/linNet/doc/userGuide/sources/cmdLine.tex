\chapter{The Command Line Interface}

\linnet{} is a simple console application. It is fully controlled by the
command line arguments and all its output is text based, either printed
to the console or written into the output files.

All interaction with \linnet{} is done from a shell window. From such a
window type \code{linNet --help} to get a short reminder on the available
command line options.

The command line may contain program options and the names of the input
files, which are always circuit netlist files. The order of options and
file names doesn't matter but usually all the file names are placed behind
all the options. If a file name could be mixed up with an option then the
double hyphen (\code{--}) can be placed on the command line: All remaining
command line arguments are considered file names regardless of how they
look like.

Circuit netlist files are named \code{*.cnl} by convention but they may
follow any other name pattern. Only the file name extension \code{*.ckt}
is reserved and must not be used.

All the information that impacts the conducted computations is placed in
the circuit netlist files; the program options only affect the generated
output. The verbositiy of the console output can be controlled, a log file
can be created and the output for Octave is an option only. The command
line arguments in detail:
\begin{itemize}
  \item \emph{-h, --help}
    \linnet{} prints a short help text and terminates again. The return
    value to the shell is 0. Can be combined with \code{-r}
  \item \emph{-r, --version}
    \linnet{} prints the software revision and terminates again. The
    return value to the shell is 0. Can be combined with \code{-h}
  \item \emph{-v LEVEL, --verbosity=LEVEL}
    The verbosity LEVEL of the application; LEVEL is an enumeration value.
    It doesn't matter whether it is written in upper or lower case
    characters. The blank between \mbox{-v} and LEVEL is optional. LEVEL
    applies in the same way to the console output and the application log
    file. In general, the same information is written to both of these
    streams. LEVEL is one out of:
    \begin{itemize}
      \item \code{DEBUG}: A lot of internal information about the progress
        of the computation is printed to the console. Among more, the
        linear equation system is printed prior to and after running the
        solver. For complex circuits this can mean many thousand lines of
        output or much time to wait.
        
        A strong recommendation is to use this level of verbosity solely
        in conjunction with the other option \code{-s}, which suppresses
        all console output. (Writing to the log file only is times faster
        then writing to the console)
        
      \item \code{INFO}: Some progress information is printed. Among more,
        the algebraic solution of the linear equation system is printed,
        which can still mean bulky output. Use this level of verbosity
        with care 
        
      \item \code{RESULT}: This is the default level of verbosity. All
        final results are printed, i.e. the requested variables in the
        frequency domain.
        
        Additionally, if a concise log format is chosen (see \code{-f}),
        which doesn't have time stamps, then \code{RESULT} will also print
        some timing information.
        
      \item \code{WARN}: The warnings are printed. Computation results can
        be found only exported as Octave code (see~\code{-o})
        
      \item \code{ERROR} The error messages are printed. Computation
        results can be found only exported as Octave code (see~\code{-o})
        
      \item \code{FATAL}: Only the fatal error messages are printed. Fatal
        errors (as opposed to the ``normal'' errors) are those, which lead
        to immediate (but still controlled) program termination; the
        out-of-memory error is a prominent example. Computation results --
        if any -- can be found only exported as Octave code (see~\code{-o})
    \end{itemize}
    Default is \code{RESULT}. Any verbosity level includes all output of
    the less verbose levels.

  \item \emph{-f FORMAT, --format-of-log-entry=FORMAT}
    Log entry format. FORMAT is an enumeration value. It doesn't matter
    whether it is written in upper or lower case characters. The blank
    between -f and FORMAT is optional. FORMAT is one out of:
    \begin{itemize}
      \item \code{raw}: Only the pure message text is printed to the
        console and the log file
        
      \item \code{short}: Each message is preceeded by a short line
        header, which contains the consumed CPU time and the severeness of
        the message
        
      \item \code{long}: Each message is preceeded by a line header, which
        contains a time stamp, the consumed CPU time and the severeness of
        the message.
    \end{itemize}
    The default format is \code{long}

  \item \emph{-s, --silent}
    Silent operation, only a greeting is emitted. Do not write progress
    information to the console, only write into the log file. To avoid
    useless runs of the application it is not allowed to choose silent
    operation if no log file is in use (see~\code{-l})
    
  \item \emph{-l[FILENAME], --log-file-name[=FILENAME]}
    This option enables the use of a log file and controls its name. No
    log file is opened if this option is not used.
    
    You may state the name of the log file but normally you won't: If the
    option is used without argument \code{FILENAME} then an appropriate
    file name is chosen by the application. It is derived from the name of
    the circuit netlist if a single input file is given and otherwise a
    generic name is chosen.
    
    There's a constraint: Either using a log file is specified with
    \code~{-l} or~\code{-s} (or \code{--silent}) is not given 
    
  \item \emph{-c, --clear-log-file}
    Clear the log file at the beginning of operation. Default is to append
    to a possibly already existing log file
    
  \item \emph{-o[DIRNAME], --Octave-output-directory[=DIRNAME]}
    This option enables the generation of Octave script code and controls
    the location. No Octave code is generated if this option is not used.

    Octave code generated from a circuit netlist will be placed into a
    folder whose name is derived from the netlist file. As many output
    folders will be created as input files are given. All of these folders
    will be placed into a common parent directory and this directory is
    designated by \code{DIRNAME}. If \code{DIRNAME} is omitted then the
    current working directory is used as common parent directory

  \item \emph{-i, --do-not-copy-common-Octave-code}
    The generated Octave code builds on some common scripts, which are
    normally copied into each of the netlist related output folders. This
    way the generated code will run just like that in Octave, without the
    need of manipulating Octave's search path. If you dislike the over and
    over copied, always identical files you can consider to install them
    once in your Octave environment and to let them no longer be produced
    by \linnet{}.
    
    If you give this option then copying the common files into each output
    folder is simply not done.
    
    This switch is relevant only if \code{-o} is also given

\end{itemize}

If the command line parser detects a problem then it tends to print the
help text for convenience. This is done into stdout. The concrete error
report is however printed into stderr. It depends on your shell in which
fashion these two text streams are separated (or interchanged). Carefully
look for the actual error message; if you are lucky it's presented in a
different color.

If \linnet{} is integrated in a scripting environment then its return
value may matter. Normally, \linnet{} returns 0 to the calling shell. If
the command line parser recognizes an error or if any input file can't be
opened or if it causes a computational error then \linnet{} returns -1.
The error cause cannot be concluded from the return value, the log file
needs to be inspected to find out.


