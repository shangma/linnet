\chapter{Math}

\section{The elimination method}
\label{secEliminationMethod}

The basic principle of the linear equation system solver is the Gaussian
elimination method. This algorithm replaces a row of coefficients of the
LES with a linear combination of this row and a reference row such that
the leading coefficient of the row becomes a null. Doing this with two
nested loops in an systematic way leads to a triangular matrix of
coefficients, that still represents an equivalent LES and the solution of
the last unknown is already figured out. (The others need additional,
similar operations.) Normally, this algorithm is applied to numerals,
which form an algebraic field. If so, the linear combination of the two
rows is the sum of the reference row times a numeric factor and the row
under progress. To compute this factor the division needs to be defined on
the coefficients and this requires field characteristics.

The symbolic solver operates on a simpler algebraic structure, a ring,
which only knows sum and product but not the division; a prominent example
of this structure are integer numbers.

If we want to apply the Gaussian method to coefficients of a ring than we
have to use another linear combination: Both combined rows are multiplied
crosswise with the leading coefficient of the other row and these two
products are then subtracted. Replacing the row under operation with this
linear combination still fulfills the requirement of shaping null values
in a controlled way and we indeed end up with a triangular matrix and a
solution for the last unknown (and could proceed to figure out the others
as in the standard case).

The problem we face is of purely practical nature: The absolute value of
the coefficients increases in strong exponential order; for our class of
coefficients this would mean an exploding increase of both memory and
computation time. The proposed way out is a known divisor $d$; given $a$
is the coefficient resulting from the previously explained linear
combination then it holds $a = d \tilde a$, where $d$ and $\tilde a$ are
existing instances of the class of coefficients. Since $d$ is identical
for all coefficients of a row we can modify the elimination rule: We
don't replace the row under progress with the results $a$ of the linear
combination so far but with the~$\tilde a$.

The known divisor $d$ is easily computed: it can be found as the last
recently computed coefficient on the diagonal of the matrix and it is one
in the very first iteration.

Some proves are given in this appendix: The main prove confirms the
existance of the factorization $a = d \tilde a$ with the proposed $d$,
which is followed by the proves of particular characteristics of the
coefficients in the given (physical) context, characteristics that
strongly determine the chosen implementation. Finally the correctness and
finiteness of the algorithm that implements the factorization is proven --
either is not apparent from the source code.

Prior to the proves we give an example of the proposed elimination method.


\subsection{Example: Elimination with integer numbers}
\label{secExampleElimMethod}

% Here's some Octave code to prove the presented equations:
% format rat
% LES=[
% -7 ,  3 ,  5 ,  7  , -6  ,  1
%  1 ,  0 , -2 , -5  ,  3  ,  0
%  0 ,  2 ,  0 , -1  ,  13 ,  0
% -3 ,  3 , -7 , -11 ,  8  , -2
% -1 , -2 ,  5 , -3  ,  7  ,  0]
% M = LES(:,1:end-1)
% Y = LES(:,end)
% X=M\Y
% M_Y=LES(:,[1:end-2 end])
% det(M)
% det(M_Y)
%
% Other example, which leads to non cancelled result (can still be cancelled by 9):
% format rat
% LES=[
% -7 ,  3 ,  5 ,  7  , -6  ,  1
%  1 ,  0 , -2 , -5  ,  3  ,  0
%  0 ,  2 ,  0 , -1  ,  13 ,  0
% -4 ,  3 , -7 , -11 ,  8  , -2
% -1 , -2 ,  5 , -3  ,  7  ,  0]
% M = LES(:,1:end-1)
% Y = LES(:,end)
% X=M\Y
% M_Y=LES(:,[1:end-2 end])
% det(M)
% det(M_Y)

Be the following LES given:

\begin{equation}
\label{eqSampleLES}
\left(
\begin{array}{ccccc|c}
$-7$ &  $3$ &  $5$ &  $7$  & $-6$  &  $1$ \\
 $1$ &  $0$ & $-2$ & $-5$  &  $3$  &  $0$ \\
 $0$ &  $2$ &  $0$ & $-1$  &  $13$ &  $0$ \\
$-3$ &  $3$ & $-7$ & $-11$ &  $8$  & $-2$ \\
$-1$ & $-2$ &  $5$ & $-3$  &  $7$  &  $0$
\end{array}
\right)
\end{equation}

\noindent
1st Elimination step, the reference row is the first one, it is combined
with rows 2\ldots5 and the known divisor is $d=1$. The reference row
itself is not touched. This yields:

\begin{displaymath}
\left(
\begin{array}{ccccc|c}
$-7$ &   $3$ &   $5$ &   $7$ &  $-6$ &   $1$ \\
 $0$ &  $-3$ &   $9$ &  $28$ & $-15$ &  $-1$ \\
 $0$ &   $0$ & $126$ & $371$ &  $63$ & $-14$ \\
 $0$ & $-12$ &  $64$ &  $98$ & $-74$ &  $17$ \\
 $0$ &  $17$ & $-30$ &  $28$ & $-55$ &   $1$
\end{array}
\right)
\end{displaymath}

\noindent
2nd elimination step, the reference row is the second one, it is combined
with rows 3\ldots5. The known divisor is the diagonal coefficient of the
1st elimination step, $d=-7$. This yields:

\begin{displaymath}
\left(
\begin{array}{ccccc|c}
$-7$ &  $3$ &   $5$    &    $7$    &  $-6$    &   $1$    \\
 $0$ & $-3$ &   $9$    &   $28$    & $-15$    &  $-1$    \\
 $0$ &  $0$ & $126/-7$ &  $371/-7$ &  $63/-7$ & $-14/-7$ \\
 $0$ &  $0$ & $-84/-7$ &   $42/-7$ &  $42/-7$ & $-63/-7$ \\
 $0$ &  $0$ & $-63/-7$ & $-560/-7$ & $420/-7$ &  $14/-7$
\end{array}
\right) =
\end{displaymath}

\begin{displaymath}
\left(
\begin{array}{ccccc|c}
$-7$ &  $3$ &   $5$ &   $7$ &  $-6$ &  $1$ \\
 $0$ & $-3$ &   $9$ &  $28$ & $-15$ & $-1$ \\
 $0$ &  $0$ & $-18$ & $-53$ &  $-9$ &  $2$ \\
 $0$ &  $0$ &  $12$ &  $-6$ &  $-6$ &  $9$ \\
 $0$ &  $0$ &   $9$ &  $80$ & $-60$ & $-2$
\end{array}
\right) \\
\end{displaymath}

\noindent
3rd elimination step, the reference row is the third one, it is combined
with rows 4 and 5. The known divisor is the diagonal coefficient of the
second elimination step, $d=-3$. This yields:

\begin{displaymath}
\left(
\begin{array}{ccccc|c}
$-7$ &  $3$ &   $5$ &    $7$    &   $-6$    &    $1$    \\
 $0$ & $-3$ &   $9$ &   $28$    &  $-15$    &   $-1$    \\
 $0$ &  $0$ & $-18$ &  $-53$    &   $-9$    &    $2$    \\
 $0$ &  $0$ &   $0$ &  $744/-3$ &  $216/-3$ & $-186/-3$ \\
 $0$ &  $0$ &   $0$ & $-963/-3$ & $1161/-3$ &   $18/-3$
\end{array}
\right) =
\end{displaymath}

\begin{displaymath}
\left(
\begin{array}{ccccc|c}
$-7$ &  $3$ &   $5$ &    $7$ &   $-6$ &  $1$ \\
 $0$ & $-3$ &   $9$ &   $28$ &  $-15$ & $-1$ \\
 $0$ &  $0$ & $-18$ &  $-53$ &   $-9$ &  $2$ \\
 $0$ &  $0$ &   $0$ & $-248$ &  $-72$ & $62$ \\
 $0$ &  $0$ &   $0$ &  $321$ & $-387$ & $-6$
\end{array}
\right) \\
\end{displaymath}

\noindent
Last elimination step, the reference row is the fourth one, it is combined
with row 5. The known divisor is the diagonal coefficient of the
third elimination step, $d=-18$. This yields:

\begin{displaymath}
\left(
\begin{array}{ccccc|c}
$-7$ &  $3$ &   $5$ &    $7$ &      $-6$            &       $1$           \\
 $0$ & $-3$ &   $9$ &   $28$ &     $-15$            &      $-1$           \\
 $0$ &  $0$ & $-18$ &  $-53$ &      $-9$            &       $2$           \\
 $0$ &  $0$ &   $0$ & $-248$ &     $-72$            &      $62$           \\
 $0$ &  $0$ &   $0$ &    $0$ & $119088/-18=-6616$ & $-18414/-18=1023$
\end{array}
\right)
\end{displaymath}

\noindent
The solution for the last unknown is $1023/-6616$. The coefficients of the
last row of the last elimination step designate the determinants of the
LES (i.e. the main determinant and the one related to the unknown
according to Cramer's rule):

\begin{displaymath}
\left|
\begin{array}{ccccc}
$-7$ &  $3$ &  $5$ &  $7$  & $-6$  \\
 $1$ &  $0$ & $-2$ & $-5$  &  $3$  \\
 $0$ &  $2$ &  $0$ & $-1$  &  $13$ \\
$-3$ &  $3$ & $-7$ & $-11$ &  $8$  \\
$-1$ & $-2$ &  $5$ & $-3$  &  $7$
\end{array}
\right| = -6616
\end{displaymath}

\begin{displaymath}
\left|
\begin{array}{ccccc}
$-7$ &  $3$ &  $5$ &  $7$  &  $1$ \\
 $1$ &  $0$ & $-2$ & $-5$  &  $0$ \\
 $0$ &  $2$ &  $0$ & $-1$  &  $0$ \\
$-3$ &  $3$ & $-7$ & $-11$ & $-2$ \\
$-1$ & $-2$ &  $5$ & $-3$  &  $0$
\end{array}
\right| = 1023
\end{displaymath}


\subsection{Prove of statement about known divisor}

The prove of the statement that we can find a common divisor in each
elimination step as the diagonal coefficient of the previous elimination
step is based on some algebraic transformations, which will show that
after the elimination step each coefficient has the meaning of the
determinant of a sub-matrix of the LES. Due to the Laplace expansion of a
determinant, which computes the determinant only with products and sums,
this must be an existing element of the class of coefficients.

% @{}: Discard indentation of before and/or behind columns
\medskip
\noindent
\begin{tabular}{p{0.91\textwidth}r}
Definition: Be $M^{[k]}_{m n}$ the sub-matrix of $M$, which
has the dimension $k \times k$ and the coefficients $a_{ij}$ of $M$ at
$i=1..k-1,m$ and $j=1..k-1,n$ with $m, n \geq k$.
& \eqnum \label{eqDefSubMatrix} \\
\end{tabular}
\smallskip

\noindent
$M^{[k]}_{m n}$ is a square matrix. It inherits the upper-left
square of $M$ and has one additional row that has the
coefficients of row $m$ of $ M$ and one additional column that
has the coefficients of row $n$ of $M$.

Examples: If $M$ is the matrix (\ref{eqSampleLES}) from the example
in section \ref{secExampleElimMethod} then we have $M^{[1]}_{1 4}=7$,
\begin{displaymath}
M^{[2]}_{2 3} =
\left(
\begin{array}{cc}
-7 &   5 \\
 1 &  -2
\end{array}
\right)
\end{displaymath}
or
\begin{displaymath}
M^{[4]}_{4 6} =
\left(
\begin{array}{cccc}
-7 &  3 &  5 &  1 \\
 1 &  0 & -2 &  0 \\
 0 &  2 &  0 &  0 \\
-3 &  3 & -7 & -2
\end{array}
\right)
\end{displaymath}

% @{}: Discard indentation of before and/or behind columns
\medskip
\noindent
\begin{tabular}{p{0.91\textwidth}r}
Lemma: Be $M$ the representation of a LES of dimension $p \times q$ with
$q \geq p$ and $M^{(k)}$ with $k=1 \ldots p-1$ the result of elimination
step $k$ of the proposed elimination method and $a^{(k)}_{m n}$ the
coefficient of $M^{(k)}$ at position $mn$. It holds:
\begin{equation*}
a^{(k)}_{m n} = \left|M^{[k+1]}_{m n}\right| \qquad \text{with } m,n > k
\end{equation*}
& \eqnum \label{eqLemmaCoefIsDet} \\
\end{tabular}
\smallskip

\noindent
Lemma (\ref{eqLemmaCoefIsDet}) is what we still have to prove: If we know
that the resulting coefficients are determinants of sub-matrices of the
original, the LES describing matrix $M$ then it must be existing
valid instances from the class of coefficients and the factorization was
possible. 

The prove of lemma (\ref{eqLemmaCoefIsDet}) is given for a particular
coefficient $a_{ij}$ of the LES. We have to consider the position of the
coefficient in the matrix and and the elimination step. The elimination
step is denoted by the top index in parenthesis. The original LES is
represented by matrix $M^{(0)}$. (Elimination step 0 means prior to the
first elimination step. $M^{(0)}$ is also simply written as $M$ -- the
explicit indication of elimination step 0 is mostly omitted.) The lemma
relates the coefficient $a^{(k)}_{ij}$ that is the result of elimination
step $k$ with the determinant of a particular, square sub-matrix of
$M^{(0)}$. The idea of the prove is to look at the coefficients of $M^{(0
\ldots k)}$ at the positions that belong to this sub-matrix and to keep
track what the operations of the eliminations steps mean to the value of
the determinant of the sub-matrix. This leads to a chained equation, which
finally proves the lemma. (Opposed to the relation of the $M^{(0 \ldots
k)}$, which is an equivalence of LESs.) This looking at the coefficients
of the sub-matrix means to reduce the prove for any elimination step of
any rectangular matrix to the final elimination result of a square
matrix.


\subsection{Reducing the problem to square matrices}
\label{secReduceToSqMat}

It is essential that all the sub-matrices $M^{[1 \ldots n](0)}_{ij}$ in
question are square and that they share the upper-left square area of $M$.
All the factors of the linear combinations of rows to be computed are
derived only from the coefficients found in this shared area and therefore
the linear combinations made on the rows of the LES are always the same as
one would have to use to perform the elimination of only the sub-matrix or
of its determinant. This is now illustrated with an example:

Be $M$ the matrix, which represents a LES with 5 unknowns and 2
knowns:
\begin{equation}
M=\left(
\begin{array}{ccccc:cc}
a_{11} & a_{12} & a_{13} & a_{14} & a_{15} & a_{16} & a_{17} \\
a_{21} & a_{22} & a_{23} & a_{24} & a_{25} & a_{26} & a_{27} \\
a_{31} & a_{32} & a_{33} & a_{34} & a_{35} & a_{36} & a_{37} \\
a_{41} & a_{42} & a_{43} & a_{44} & a_{45} & a_{46} & a_{47} \\
a_{51} & a_{52} & a_{53} & a_{54} & a_{55} & a_{56} & a_{57} \\
\end{array}
\right)
\end{equation}

\noindent
Let's take the coefficient at row 5, column 4: The lemma says

\begin{equation*}
a^{(1)}_{54} = \left| M^{[2]}_{54} \right| =
\left|
\begin{array}{cc}
a_{11} & a_{14} \\
a_{51} & a_{54}
\end{array}
\right|
\end{equation*}

\noindent
When the first iteration of the elimination of the matrix $M$ reaches the
last row it replaces:

\begin{equation*}
R^{(1)}_{5} = a_{11} R^{(0)}_{5} - a_{51} R^{(0)}_{1} \text{,}
\end{equation*}

\noindent
which is -- for the affected four coefficients shared between $M$ and
$M^{[2]}_{54}$ -- exactly the same operation as an elimination of
$M^{[2]}_{54}$ would be:

\begin{equation*}
M^{[2](1)}_{54} =
\left(
\begin{array}{cc}
a_{11} & a_{14} \\
0      & a_{11} a_{54} - a_{51} a_{14}
\end{array}
\right)
\end{equation*}

\noindent
All other operations of the first iteration of the
elimination on $M$ affect rows and columns outside of $M^{[2]}_{54}$ and have no
impact on any of the coefficients of $M^{[2](1)}_{54}$.

Or let's take the coefficient at row 4, column 6: The lemma says

\begin{equation*}
a^{(2)}_{46} = \left| M^{[3]}_{46} \right| =
\left|
\begin{array}{ccc}
a_{11} & a_{12} & a_{16} \\
a_{21} & a_{22} & a_{26} \\
a_{41} & a_{42} & a_{46}
\end{array}
\right|
\end{equation*}

\noindent
(Please note, that this example refers to a coefficient,
which has not the final value of the complete elimination, there will also
be the value $a^{(3)}_{46}$ during further processing.)

In the first elimination step and when the process reaches rows 2 and 4
then shared coefficients are touched. The factors of the linear
combination of rows ($(a_{11}, a_{21})$ and $(a_{11}, a_{41})$ for rows 2
and 4, respectively) are located at according positions in $M$ and
$M^{[3]}_{46}$ (i.e. in the first column) and thus the elimination of $M$
is exactly the same operation as seen in its sub-matrix $M^{[3]}_{46}$.
For both matrices the identical row replacements

\begin{eqnarray*}
R^{(1)}_{2} &=& a_{11} R^{(0)}_{2} - a_{21} R^{(0)}_{1} \\
R^{(1)}_{4} &=& a_{11} R^{(0)}_{4} - a_{21} R^{(0)}_{4}
\end{eqnarray*}

\noindent
take place (using the same global position indices in the matrix
representations).

If we look at the second elimination step it's just the same: the operation
performed on the rows of matrix $M^{(1)}$ is

\begin{equation*}
R^{(2)}_{i} = a^{(1)}_{22} R^{(1)}_{i} - a^{(1)}_{i2} R^{(1)}_{2}
\qquad \text{with } i=3 \ldots 5
\end{equation*}

\noindent
Coefficients, which are shared with the sub-matrix are affected only for
$i=4$ and then we see the same operation on the coefficients of the
sub-matrix that would be used to run the second elimination step directly
on the isolated sub-matrix. The required factors $(a^{(1)}_{22},
a^{(1)}_{42})$ of the linear combination are found at the appropriate
position in the sub-matrix.

This idea holds in general for all sub-matrices in question because they
all share their $k-1$ top-most rows and left-most columns with the
original matrix $M$ and the factors of all the linear combinations are
taken from these rows and columns - be it the original coefficients or the
products of the $k-2$ involved, first elimination steps.

This is why we can now reduce the prove to the elimination of the square
sub-matrices. We have to show that the final value of the bottom-right
coefficient in the square area is identical to the determinant of the
original sub-matrix. This result for a particular but arbitrary
coefficients can then be applied to any coefficient computed in any
elimination step.

But before we can start with the prove we need to introduce a minor lemma,
which is applied in the following.


\subsection{Linear combination of rows of determinant}

For the following steps we need to understand how the value of the
determinant of a square matrix is changed if a row of the matrix is
replaced by the linear combination of itself and another row.

% @{}: Discard indentation of before and/or behind columns
\medskip
\noindent
\begin{tabular}{p{0.91\textwidth}r}
Lemma: Be $M$ a square matrix and $\hat M$ the same matrix, where only row
$R_j$ has been replaced with the linear combination $c_j R_j +
c_i R_i$ of this row and the other row $R_i$. It holds: $|\hat M|=c_j
|M|$.
& \eqnum \label{eqLemmaLinCombRowsD} \\
\end{tabular}
\smallskip

\noindent
To prove lemma (\ref{eqLemmaLinCombRowsD}) we just need to expand $|\hat
M|$ along its row $j$:

% =
%c_j \left|
%\begin{array}{ccc}
%a_{11} & \ldots & a_{1n} \\
%\vdots & \ddots & \vdots \\
%a_{n1} & \ldots & a_{nn}
%\end{array}
%\right| =
%c_j \left| M \right|

\begin{eqnarray}
\left| \hat M \right| & = &
\left|
\begin{array}{ccc}
a_{11}                  & \ldots & a_{1n} \\
\vdots                  &        & \vdots \\
a_{i1}                  & \ldots & a_{in} \\
\vdots                  &        & \vdots \\
c_j a_{j1} + c_i a_{i1} & \ldots & c_j a_{jn} + c_i a_{in} \\
\vdots                  &        & \vdots \\
a_{n1}                  & \ldots & a_{nn}
\end{array}
\right| \nonumber \\
  & = & \sum^{n}_{c=1}(-1)^{i+c} (c_j a_{jc} + c_i a_{ic}) D_{jc}
\qquad \text{with } D_{ij} \text{ being the sub-determinant at position } ij \nonumber \\
  & = & c_j \sum^{n}_{c=1}(-1)^{i+c} a_{jc} D_{jc}
        + c_i \sum^{n}_{c=1}(-1)^{i+c} a_{ic} D_{jc} \label{eqProveLemmaLinCombRowsD} \\
  & = & c_j \left| M \right| + c_i \cdot 0 \nonumber
\end{eqnarray}

\noindent
Both determinants in lemma (\ref{eqLemmaLinCombRowsD}), $|M|$ and $|\hat
M|$, share all rows but row $j$, hence all sub-determinants $D_{jc}, c=1
\ldots n$ are identical for both. Therefore the sum in the first addend in
(\ref{eqProveLemmaLinCombRowsD}) is the Laplace expansion of the
determinant of $M$. The sum in the second addend in
(\ref{eqProveLemmaLinCombRowsD}) is null; it is the Laplace expansion of a
similar determinant: all rows but $j$ are identical but row $j$ is
replaced by row $i$. The determinant of a matrix with two identical thus
linear dependent rows is null. Equation (\ref{eqProveLemmaLinCombRowsD})
reduces to lemma (\ref{eqLemmaLinCombRowsD}).

% Determinant with identical rows is null: Swap two rows: Expansion
% according to sum of all permutated tupels proves that the sign of the
% determinant is inverted. If these rows are identical then the
% determinant has not changed at all, so the value is unchanged. This can
% only hold for the particular value null.


\subsection{Prove for square sub-matrix}

From now on we take the view at the sub-matrix only. To simplify the
explanations we change the position indices of the coefficients. We are
going to use consecutive one based indices as usual. The positional
relationship with the matrix of the LES might be obscured but is out of
scope in this section anyway.

We investigate the elimination of an arbitrary square sub-matrix $S$
according to definition (\ref{eqDefSubMatrix}) of $M$, the matrix describing
the original LES:

\begin{eqnarray*}
S^{(0)} & = & M^{[n](0)}_{ij}\qquad \text{with } n>1, i \geq n, j \geq n \\
        & = &
\left(
\begin{array}{ccc}
a_{11} & \ldots & a_{1n} \\
\vdots & \ddots & \vdots \\
a_{n1} & \ldots & a_{nn}
\end{array}
\right) \qquad \text{(here using the renamed position indices)}
\end{eqnarray*}

\noindent
and define $D$ to be its determinant:

\begin{eqnarray*}
D & := & \left| S^{(0)} \right| \\
  & =  &
\left|
\begin{array}{ccc}
a_{11} & \ldots & a_{1n} \\
\vdots & \ddots & \vdots \\
a_{n1} & \ldots & a_{nn}
\end{array}
\right|
\end{eqnarray*}

\noindent
Please note that due to the redefinition of the position indices $a_{ij}$
no longer is the coefficient of $M$ in row $i$ and column $j$ if either
$i=n$ or $j=n$.

After the first elimination step of the LES -- and according to section
\ref{secReduceToSqMat} the first elimination step of $S$ -- the matrix and
its determinant change as follows:

\begin{eqnarray}
S^{(1)} & = &
\left(
\begin{array}{cccc}
a_{11} & \ldots       &           & a_{1n}       \\
0      & a^{(1)}_{22} & \ldots    & a^{(1)}_{2n} \\
\vdots & \vdots       & \ddots    & \vdots       \\
0      & a^{(1)}_{n2} & \ldots & a^{(1)}_{nn}
\end{array}
\right) \nonumber \\
D^{(1)} & := & \left| S^{(1)} \right|                                 \nonumber \\
        & =  & \frac{\left(a_{11}\right)^{(n-1)}}{1^{(n-1)}} \left| S(0) \right| \nonumber \\
        & =  & \frac{\left(a_{11}\right)^{(n-1)}}{1^{(n-1)}} D \label{eqSqrMatD1}
\end{eqnarray}

\noindent
The factor in front of $D$ in equation (\ref{eqSqrMatD1}) is explained by
the $n-1$ row operations that form the elimination step: The rows $2
\ldots n$ are all multiplied by $a_{11}$ as part of the linear combination
with the first row, which explains according to lemma
(\ref{eqLemmaLinCombRowsD}) the numerator of the factor. Then the rows are
all divided by the common divisor, which is one in the first elimination
step. This explains the denominator of the factor.

If we now present the second elimination step, too, then a general pattern
becomes apparent, which gives us the wanted result for any size of matrix
and its determinant:

\begin{eqnarray}
S^{(2)} & = &
\left(
\begin{array}{ccccc}
a_{11} & \ldots       &              &        & a_{1n}       \\
0      & a^{(1)}_{22} & \ldots       &        & a^{(1)}_{2n} \\
0      & 0            & a^{(2)}_{33} & \ldots & a^{(2)}_{3n} \\
\vdots & \vdots       & \vdots       & \ddots & \vdots       \\
0      & 0            & a^{(2)}_{n3} & \ldots & a^{(2)}_{nn}
\end{array}
\right) \qquad \text{with }
a^{(2)}_{ij} = \frac{a^{(1)}_{22} a^{(1)}_{ij} - a^{(1)}_{i2} a^{(1)}_{2j}}{a_{11}} \\
\nonumber \\
D^{(2)} & = & \frac{\left(a^{(1)}_{22}\right)^{(n-2)}}{(a_{11})^{(n-2)}} D^{(1)} \nonumber \\
        & = & \frac{\left(a^{(1)}_{22}\right)^{(n-2)}}{(a_{11})^{(n-2)}}
              \frac{\left(a_{11}\right)^{(n-1)}}{1^{(n-1)}} D                    \nonumber \\
        & = & (a^{(1)}_{22})^{(n-2)} a_{11} D \label{eqSqrMatD2}
\end{eqnarray}

\noindent
The pattern: In elimination step $k$ we multiply $n-k$ rows with the
value $v = a^{(k-1)}_{kk}$, which temporarily and according to lemma
(\ref{eqLemmaLinCombRowsD}) increases the determinant of the modified
matrix by $v^{n-k}$ but in the next elimination step $k+1$, when $n-k-1$
rows are processed the same value $v$ is used as common divisor, which
decreases the determinant again by $v^{n-k-1}$ -- and a simple v to the
power of one remains as additional factor in the equation between
$D^{(k)}$ and $D$. This leads to the final elimination result, the result
of elimination step $n-1$:

\begin{eqnarray}
S^{(n-1)} & = &
\left(
\begin{array}{ccccc}
a_{11} & \ldots       &              &        & a_{1n}       \\
0      & a^{(1)}_{22} & \ldots       &        & a^{(1)}_{2n} \\
0      & 0            & a^{(2)}_{33} & \ldots & a^{(2)}_{3n} \\
\vdots & \vdots       &              & \ddots & \vdots       \\
0      & 0            & \hdots       &        & a^{(n-1)}_{nn}
\end{array}
\right) \nonumber \\
D^{(n-1)} & = & \prod^{n-1}_{i=1} a^{(i-1)}_{ii} D \label{eqSqrMatDNm1}
\end{eqnarray}

\noindent
$S^{(n-1)}$ is a triangular matrix. Its determinant is the product of all
the diagonal elements. This product can be equated with
(\ref{eqSqrMatDNm1}). Please note, that the expressions in parenthesis
denote the elimination step not a power:

\begin{equation*}
\left| S^{(n-1)} \right| = \prod^{n}_{i=1} a^{(i-1)}_{ii}
                         = D^{(n-1)}
                         = \prod^{n-1}_{i=1} a^{(i-1)}_{ii} D
\end{equation*}

\noindent or

\begin{equation}
a^{(n-1)}_{nn} = D
\end{equation}

\noindent
This is what we wanted to prove; the last diagonal element after the last
elimination step has the value $D$ of the determinant of the original
matrix $S$.

The prove holds for all $n>1$ and any coefficient.\footnote{And for $n=1$
by definition, too, if we consider the original matrix the result of the
``0-th'' elimination step.} All coefficients at any position and after any
elimination step of the matrix $M$, which represents the LES have the
meaning of the determinant of the one or other sub-matrix of the
\emph{original} matrix $M$.

All the appearing determinants and thus all ever appearing coefficients
can be represented as a sum of products of original coefficients. This
defines the arithmetic class of the coefficients and their implementation
with respect to storage and needed operations. To perform the elimination
step we need to conduct the linear combination of a pair of coefficients
(the linear combination of rows as used above is broken down to
coefficients via an iteration along the rows), followed by the
factorization with respect to the known divisor. Both together is called
the ``elementary step'' in the implementation. The elementary step can be
further broken down into the operations product, sum and factorization. A
real division is not required.

In the particular use case of \linnet{} the original coefficients are
derived from physical facts that bear specific properties, which further
refines the definition of their arithmetic class; this is explained in
section \ref{secProveOfNonQuadraticCoef} and the following.


\subsection{Pivoting}

The linear combination used in an elimination step must never multiply the
destination row with null; all the information of the related equation
would be discarded and the final solution could not be figured out.
Therefore, an implementation of the Gauss elimination method uses pivoting
and so does \linnet{}'s extended method. If it holds $a^{(k-1)}_{kk} = 0$
for the diagonal coefficient at the beginning of elimination step $k$ then
the algorithm looks for a coefficient $a^{(k-1)}_{ck} \neq 0$ with $c >
k$. If there is such a coefficient then the related row is swapped with
row $k$. If there is no such coefficient then the elimination has
completed with the result that there is no unambiguous solution for the
unknowns of the LES. Because of the immediate abortion of the elimination
the latter case has no impact on our proves. But even the former case is
irrelevant because the instant row swapping ``in case'' is identical to
the same (hypothetic) swapping of rows prior to the complete elimination
process. ``Identical'' means that all chosen linear combinations, all
common divisors and all resulting coefficients in all elimination
steps would be just the same. And since the proves hold for any LES and
matrix it also holds for the (hypothetic) LES and matrix with rows
swapped prior to the elimination.


\subsection{Computational effort}

The order of complexity of the computation has not been investigated in
depth. Obviously, as we use a modified Gauss elimination, it is $O(n^3)$
in terms of elementary steps to bring the matrix in triangular shape.
However, different to the Gauss elimination method there's no inverse
elimination to bring it into diagonal form. The proposed method requires
$n$-fold elimination for the $n$ unknowns, which means a complexity of
$O(n^4)$ in terms of elementary steps for a full solution of the LES.

In the \linnet{} use case the latter is irrelevant as the number of
unknowns of interest, i.e. the really figured out unknowns, does typically
not depend on the size of the circuit or the total number of unknowns; for
this use case with a rather constant number of computed unknowns $O(n^3)$
still holds.

The bigger problem is the elementary step itself. It is the atomic operation, that
combines a four-tuple of coefficients to a new coefficient. This operation
is far from constant; the more the algorithm advances the longer the
coefficients become and the longer this operation takes. Finally this
operation yields the wanted result. In theory an upper boundary of this
operation would be $O(n!)$; in a fully interconnected network, where each
node is connected to each other node we will need to have $n!$ independent
terms in the resulting expressions and these will need an according amount
of time to be figured out. For practical relevant use cases full
interconnectivity won't ever occur, sparse matrices are typical. Which makes
it better but the order will still be far from good.

A possible approach is the simplification that a network has a (more or
less) constant number $c$ of interconnections (i.e. devices) between its
nodes. If we furthermore disregard terms, which eliminate each other in
the computed products of coefficients then the lengths of the coefficients
were taken to the power of two in each elimination step and when using the
unmodified Gauss elimination with linear combinations of rows. The
length at the beginning is $c$. Repeatedly increasing this length leads to
the complexity of $O(c^{(2^{n-1})})$.

Due to the applied common divisor the proposed elimination method
significantly reduces the growth in length of the coefficients. Across the
eliminations steps $k=1 \ldots (n-1)$ the length will now grow like $c^2, c^3
= \frac{(c^2)^2}{c}, c^4 = \frac{(c^3)^2}{c^2}, \ldots$ This leads to a
complexity of still $O(c^n)$.

% TODO Consider to try to do the sum: c^2*(n-1)^2 +
% c^4*(n-2)^2 + c^8*(n-3)^2 etc. and c^2*(n-1)^2 + c^3*(n-2)^2 +
% c^4*(n-3)^2 etc.
Combining these considerations with the complexity of the elimination
pattern, i.e. considering the quadraticly decreasing number of affected
coefficients, could be tried to refine the result but the complexity of
the elementary step will anyway be dominant.

The improvement of the proposed factorization is illustrated by a numeric
example. If we put $c=3$ and $n=8$ then $O(c^{(2^{n-1})})$ means about
$10^{61}$ units of effort and $O(c^n)$ only about 6500 units, units
however, which are significantly more expensive in terms of computational
effort due to the factorization.

The lengths of the coefficients and the computational effort for the
elementary steps are closely related as linear lists are applied. The
order of effort in terms of memory usage will be the same as for
computational effort. The main impact on the computational effort is the
growing length of the coefficients and the length of the coefficients is
proportional to the memory consumption.

Since the growth in length of the coefficients seems to have the
overwhelming impact on memory and computational effort a possible
optimization idea is an extended pivoting. Rather than looking for the
first non null coefficient we could look for the row with the shortest
coefficients. (If the longer rows are processed later than far less
coefficients are affected.) However this kind of pivoting would in turn
bring a significant additional complexity as it in turn depends on all
(remaining) coefficients and their lengths. The next problem would be the
definition of ``row with the shortest coefficients'': which norm to be
applied; the average length is not necessarily the right choice.

One has to see that the complexity of the duration of the computation is
in line with the complexity of the eventually yielded result: Given you
were very patient and would wait a number of years for the result
then you'd probably be frustrated by this result. Kilometers of paper to
print the formula and no practical use at all. As a matter of experience
we can state that the analysis of circuits that have meaningful and still
manageable transfer functions is uncritical with respect to computational
effort.

In this sense the theoretically poor order of the algorithm is harmless
with respect to the imaginable practical use cases of the software.


\subsection{Common divisors in final solution}

The proposed elimination does not guarantee that the found solution (i.e.
the found numerator and denominator for the right-most unknown) is free of
common divisors. It can only guarantee that it won't introduce additional
common divisors as the unmodified Gaussian elimination without the
factorization would. If the method is e.g. applied to a matrix of
integers, where all coefficients of a row or column have a common divisor
then this common divisor will be propagated into the final result.

% The statement must not be inverted: If all rows and columns are free of
% common divisors then this doesn't guarantee that the result is free of
% a common divisor; an according example exists for integer numbers (see
% above in section "Example: Elimination with integer numbers"). This is
% probably an integer effect, where sums can lead to new factors, e.g. the
% two prime numbers 5 and 7 have the factorisable sum 12. For our
% unrelated symbols this effect is not present: s+t is generally not
% factorisable if s and t are atoms.

In our application of the method the initial coefficients of the matrix
describe independent and unrelated physical devices. The elimination
method itself doesn't introduce common divisors, which means that the
final result should always be fully canceled.

For explanatory purpose it has to be mentioned that the relations between
the values of devices of same kind, which can be stated in the netlist
file are completely irrelevant to the elimination process of the LES.
These relations are simply not known by the solver. They are applied only
in the final rendering of the found solution and respecting the wanted
absence of common divisors in the rendered result becomes a simple matter
of traditional computation of the greatest common divisor of all affected
numeric factors.


\section{Prove of non quadratic coefficients}
\label{secProveOfNonQuadraticCoef}

The elimination method as pointed out in section
\ref{secEliminationMethod} will work with any kind of objects of kind
ring. The implementation of the solver will however strongly depend on the
specific kind of object. The basic operations store, add and multiply and
the factorization $a = d \tilde a$ need to be implemented. Our kind of
objects, i.e. our kind of matrix coefficients, has some particular
characteristics, which are strongly exploited in the implementation.

Section \ref{secModellingDevices} on page \pageref{secModellingDevices}
explains how the LES is set up. The modeling of the electronic devices
determines how the coefficients of the LES look like (prior to the
elimination). Summarizing this section we can state: At the beginning, all
coefficients are sums of symbols and/or the number one. The addends of
these sums may have both signs. No products of symbols appear and in
particular no symbol is taken to a power other than one. No two
addends of a coefficient are identical, so there's no possibility to
agglomerate symbols leading to a product of a symbol and a number other
than $\pm 1$.

During the elimination steps the initial coefficients are multiplied and
added and factorized. The resulting coefficients will still be sums. The
addends will now either be products of different symbols and $\pm 1$ or
the number $\pm 1$. The further characteristics of the initial
coefficients (symbols taken to the power of one, no identical addends
leading to products of symbols and a number other than $\pm 1$) are
retained. The coefficients can thus be represented by a list of signed
products of symbols. Each known symbol is either present in the product
(power of one) or not (power of null). The product can be represented by a
bit vector, where each bit is related to one known symbol. The sign of an
addend could be represented by a Boolean, but actually isn't: The numeric
factor in an addend is $\pm 1$ for a coefficient prior to and after each
elimination step but can take (small) integer values in intermediate
computation results. To avoid different representations of expressions
during the computation and in the computation result we use integer
numbers in general to represent the numeric factor.

While the characteristics of the initial coefficients result from the
modelling of the devices in the setup phase of the LES is the preservation
of these characteristics during the elimination far from self-explanatory.
This one and the next sections give the required proves. In the first step it
is proven that the symbols in the products in the addends will never
appear with a power other than one or null.

% @{}: Discard indentation of before and/or behind columns
\medskip
\noindent
\begin{tabular}{p{0.9\textwidth}r}
Lemma: The value of the determinant of any of the sub-matrices of the LES
according to definition (\ref{eqDefSubMatrix}) can be brought into the
form of a sum of products of a numeric factor $f$ and all the device
symbols taken to a power of either null or one. In this form no two
addends of the sum have the same set of symbol powers.
& \eqnum \label{eqLemmaNoQuadratic} \\
\end{tabular}
\smallskip

\noindent
Since all in the course of the elimination ever appearing coefficients
have the value of one of these subdeterminants the same statement holds
for all coefficients of the matrix after all the elimination steps.

The device symbols are the given R, L, C, etc. They appear in the addends
of the coefficients (power = 1) or they don't appear (power = 0) but --
and this is the gist of the lemma -- they will never appear with a
negative power or with a power of two or higher. To put it in an example:
$R_1 R_2 C + R_1 L$ could be a coefficient but $R_1 R_1 C + R_1 L$ or
$\frac{R_1}{R_2} C + R_1 L$ are expressions, which will definitely never
appear as a coefficient of the matrix after an elimination step.

The importance of the lemma is evident: The addends of the coefficients
can be implemented as bit vectors and their products become cheap bit-wise
integer operations.

The prove of lemma (\ref{eqLemmaNoQuadratic}) exploits that there are
symmetries of the symbols in the LES. Any symbol $s$ always appears in a
characteristic pattern:
\begin{equation}
\label{eqMatrixPatternInS}
\left(
\begin{array}{ccccc}
\ddots & & & & \\
 & \pm s & \ldots & \mp s & \\
 & \vdots & & \vdots & \\
 & \mp s & \ldots & \pm s & \\
 & & & & \ddots \\
\end{array}
\right)
\end{equation}

The explanation of the pattern is the meaning of symbol $s$: It designates
the current flow through device $s$ from one node of the network to
another one\footnote{In the internal representation of the network all
devices have the meaning of conductance, i.e. the complex proportionality
factor between current and voltage.}; and the current is the same at both
ends. The sign of a current is defined from the node's perspective:
Effluent is a negative current influent is positive by definition. A row
of the LES is the current balance of a node, the columns are related to
the unknown node's voltage potentials. An increase of the own potential
will lead to an affluent current (thus negative sign), whereas an increase
of the potential of the far end's node will lead to an influent current
with positive sign. This leads to the pair $(+s, -s)$ in a row of the
LES. The other row containing the inverse pair results accordingly from
the current balance of the far end's node. Here the pair appears inverse
as we take the perspective of the other node.

Two exceptions have to be considered. We always have a ground node, which
no current balance is made for, and $s$ might be connected to this node. In
which case one of the two rows would disappear. Furthermore, the ground
node has null potential by definition, there's no related unknown or
column. One of the two columns disappears. $s$ appears only once in the
LES with negative sign.

The controlled sources introduce coefficients $s$ which also appear in
pairs, but here we have the additional freedom that the source is connected
two one or two non ground nodes and the control input is connected to one
or two non ground nodes. Consequently, the disappearing row and the
disappearing column can happen independently resulting in either two
pairs, or a vertical pair only or a horizontal pair only or a single
appearance of $s$ (the latter here with positive sign). Summarizing we can
observe only the following patterns of appearance of $s$ in the LES
(simplified representation, disregarding other symbols and coefficients):
\begin{equation}
\label{eqPatternsOfAppearance}
\begin{array}{cccc}
\left(
\pm s
\right)
&
\left(
\begin{array}{cc}
\pm s & \mp s
\end{array}
\right)
&
\left(
\begin{array}{c}
\pm s \\
\mp s
\end{array}
\right)
&
\left(
\begin{array}{cc}
\pm s & \mp s \\
\mp s & \pm s \\
\end{array}
\right)
\end{array}
\end{equation}

It has been proven that all appearing coefficients have the meaning of the
determinant of a sub-matrice of the LES, see lemma (\ref{eqLemmaCoefIsDet}).
All such sub-matrices retain the same structural characteristics of the
appearance of $s$; by extracting a subdeterminant from the matrix a row
or a column with $s$ might disappear or $s$ might completely disappear,
but we will never see a new pattern for the remaining symbols in the
subdeterminant.

According to lemma (\ref{eqLemmaLinCombRowsD}) the value of a determinant
is not changed if we replace a row or column by the sum of this and
another row or column, respectively. Applying this operation first to the
two columns that contain the symbol and then to the two rows that contain
the symbol, we can always reduce all possible patterns to the simple
pattern of having the symbol just once in the determinant. (Obviously, it
doesn't matter if we skip one or both of theses steps in case $s$ appears
in one of the simpler patterns.)

The coefficients of the LES are sums of differing symbols. A coefficient
can thus be written as the symbol of interest, $s$, plus a remainder, an
arbitrary sum of \emph{other} symbols: $a_{ij} = \pm s + R_{ij}$. We can
modify the determinant of the sub-matrix as follows:

\begin{displaymath}
\left|
\begin{array}{ccccc}
\ddots & & & & \\
 & \pm s + R_{iu} & \ldots & \mp s + R_{iv} & \\
 & \vdots & & \vdots & \\
 & \mp s + R_{ju} & \ldots & \pm s + R_{jv} & \\
 & & & & \ddots \\
\end{array}
\right| = \ldots
\end{displaymath}

\begin{displaymath}
\left|
\begin{array}{ccccc}
\ddots & & & & \\
 & \pm s + R_{iu} & \ldots & R_{iu} + R_{iv} & \\
 & \vdots & & \vdots & \\
 & \mp s + R_{ju} & \ldots & R_{ju} + R_{jv} & \\
 & & & & \ddots \\
\end{array}
\right| = \ldots
\end{displaymath}

\begin{equation}
\label{eqUnificationInSOfSubD}
\left|
\begin{array}{ccccc}
\ddots & & & & \\
 & \pm s + R_{iu}  & \ldots & R_{iu} + R_{iv}                   & \\
 & \vdots & & \vdots & \\
 & R_{iu} + R_{ju} & \ldots & R_{iu} + R_{iv} + R_{ju} + R_{jv} & \\
 & & & & \ddots \\
\end{array}
\right|
\end{equation}

The modifications of the determinant are shown for the case if
$s$ appears in the most complex appearance pattern. For the simpler
patterns less modification steps are required but the final form of
Equation (\ref{eqUnificationInSOfSubD}) with a single $s$ will be the same.

Only one statement about the $R_{ij}$ is important in this context: They do
not contain any appearance of symbol $s$.

To compute the value of the determinant, we perform the Laplace expansion
across row $i$, which still contains the one and only appearance of $s$.
If we are in column $n\not=u$ the coefficient visited in row $i$ doesn't
contain $s$ and nor do the subdeterminants we have to multiply with. If we
are in column $u$ of row $i$ then we have to take the coefficient $s +
R_{iu}$ times a subdeterminant, which doesn't contain $s$. $R_{iu}$ also
has no $s$, thus the only term of the product containing $s$ is $s$ times
the $s$-free subdeterminant, which can only lead to a term of $s$ to the
power of 1. The further expansion means to sum up all such terms; this
operation doesn't create any higher power of $s$.

It is proven, that $s$ can't appear in the determinant of any sub-matrix
with a power higher than 1; it might not appear at all, which means power
of 0. Since $s$ was an arbitrary chosen symbol out of the set of all
symbols the prove holds for all the symbols.


\section{Prove of numeric factor one in coefficients' addends}
\label{secProveOfFacOne}

According to lemmas (\ref{eqLemmaCoefIsDet}) and
(\ref{eqLemmaNoQuadratic}) all coefficients possibly appearing as result
of an elimination step can be represented as a sum of products of a
numeric factor and a subset of the device symbols. Here, we will give the
prove that this factor can only have the values plus and minus one. We
show, that the expansion of the determinant of any of the sub-matrices of
the LES only has addends, which differ (i.e. differ in the tuple of
multiplied symbols). Since all addends (whose sum is the value of the
determinant) are different in this sense, no pair of two of those can be
gathered together, which would yield a numeric factor other than absolute
one.

% @{}: Discard indentation of before and/or behind columns
\medskip
\noindent
%\begin{flushleft}
\begin{tabular}{p{0.9\textwidth}r}
Lemma: The numeric factor $f$ mentioned in lemma (\ref{eqLemmaNoQuadratic})
has a value of either one or minus one.
& \eqnum \label{eqLemmaFactorOne} \\
\end{tabular}
%\end{flushleft}
\smallskip

\noindent
In the first step, this is shown for determinants of sub-matrices solely
containing null values and sums of symbols in the possible appearance
patterns according to equation (\ref{eqPatternsOfAppearance}).

The prove is recursive, we start with a determinant of dimension two. An
example of such a subdeterminant is given as illustration. It contains
five symbols, using all possible appearance patterns at least once:

\begin{displaymath}
\left|
\begin{array}{cc}
s+a-b-d   &   -s+b+d  \\
-s-a+d    &   s+c-d
\end{array}
\right|
\end{displaymath}

We will show for the particular symbol $s$ that there will be no two
identical terms in the expansion of the determinant, which contain $s$.
First we replace the other symbols from the example with a general form,
the remainders $R_{ij}$, the sums of all other symbols, and bring the
determinant in the equivalent form with identic value but a single
appearance of $s$ only; the same has been done in detail in section
\ref{secProveOfNonQuadraticCoef}:

\begin{displaymath}
\left|
\begin{array}{cc}
\pm s + R_{11}  & R_{11} + R_{i2}                   \\
R_{11} + R_{21} & R_{11} + R_{12} + R_{21} + R_{22} \\
\end{array}
\right|
\end{displaymath}

If we expand this determinant it is evident that the only term that
contains $s$ is the product $\pm s(R_{11} + R_{12} + R_{21} + R_{22})$.
The sum of the four remainders will not contain identical addends: It are
all four initial remainders, which means that for any symbol holds that
all its occurrences according to the patterns of equation
(\ref{eqPatternsOfAppearance}) are summed up. Depending on the appearance
pattern, the symbol is either eliminated or it appears just once. In the
previous example we would see $s(a-b-d + b+d + (-a+d) + c-d) = sc$ but
nothing like $s(a+a) = 2sa$. This is what we wanted to prove.

It's not obvious but trivial to show that the same holds if $s$ has one of
the other possible appearance patterns in the initial form of the
determinant. We always end up with $s$ times a sum of the $R_{ij}$ as only
addend containing $s$. The sum of the $R_{ij}$ either is a single $R_{ij}$
or the sum of a pair of two neighboured $R_{ij}$. In both cases the sum
can't contain multiple occurances of any symbol since symbols appearing in
both $R_{ij}$ will eliminate each other.

Since $s$ was an arbitrarily chosen symbol the same holds true for all
other symbols as well.

The recursion from dimension $N-1$ to $N$ is done with the Laplace
expansion of the determinant. Again, we look at the particular symbol $s$
of the determinant of an arbitrary sub-matrix of dimension $N$ of the LES.
In the first step we do the same transformation as shown in
(\ref{eqUnificationInSOfSubD}) and get a determinant of identical value
and with a single occurrence of $s$.

The transformation doesn't destroy the appearance pattern of any symbol $t
\not= s$ in a relevant way. An example, using two symbols $s$ and $t$:

\begin{displaymath}
\left|
\begin{array}{ccccccc}
\ddots &               &        &        &        &        &        \\
       & \pm s + \pm t & \ldots & \mp t  & \ldots & \mp s  &        \\
       & \vdots        &        & \vdots &        & \vdots &        \\
       & \mp s         & \ldots &        & \ldots & \pm s  &        \\
       & \vdots        &        & \vdots &        & \vdots &        \\
       &         \mp t & \ldots & \pm t  & \ldots &        &        \\
       &               &        &        &        &        & \ddots \\
\end{array}
\right| = \ldots
\end{displaymath}

\begin{displaymath}
\left|
\begin{array}{ccccccc}
\ddots &               &        &        &        &        &        \\
       & \pm s + \pm t & \ldots & \mp t  & \ldots & \pm t  &        \\
       & \vdots        &        & \vdots &        & \vdots &        \\
       & \mp s         & \ldots &        & \ldots &        &        \\
       & \vdots        &        & \vdots &        & \vdots &        \\
       &         \mp t & \ldots & \pm t  & \ldots & \mp t  &        \\
       &               &        &        &        &        & \ddots \\
\end{array}
\right| = \ldots
\end{displaymath}

\begin{equation}
\label{eqDetWithSingleSAndTheTs}
\left|
\begin{array}{ccccccc}
\ddots &               &        &        &        &        &        \\
       & \pm s + \pm t & \ldots & \mp t  & \ldots & \pm t  &        \\
       & \vdots        &        & \vdots &        & \vdots &        \\
       &         \pm t & \ldots & \mp t  & \ldots & \pm t  &        \\
       & \vdots        &        & \vdots &        & \vdots &        \\
       &         \mp t & \ldots & \pm t  & \ldots & \mp t  &        \\
       &               &        &        &        &        & \ddots \\
\end{array}
\right|
\end{equation}

\noindent
Although we see a new appearance pattern for $t$ in the transformed
determinant this is not relevant. In the next step we expand the
determinant along the row, which still contains $s$. All subdeterminants
can't contain $s$. The only terms in the result, which might contain
symbol $s$ are got, when we reach the column with $s$ and now the
subdeterminant (whose value has to be taken times $s$) again has the
required appearance pattern for $t$: The additional occurrences of $t$ do
not belong to this subdeterminant. The subdeterminant exposes the assumed
appearance pattern for all its symbols and symbol $s$ doesn't belong to
these symbols.

What has been shown for our example $t$ holds in general: The two
operations copy the symbols, which are found in row and column of the
finally remaining $s$ to another row and column. The symbols at the origin
of this copying will never be part of the subdeterminant, so there are at
maximum two new pairs of the copied symbol in the relevant subdeterminant.
If the other symbol $t$ has a simpler appearance pattern or if it shares
more positions with $s$ as in the example then there will be only one pair
or even only a single appearance of $t$ in the subdeterminant due to
elimination.

The only addends in the final expansion that contain $s$ result from the
product of $\pm s$ with the value of this subdeterminant; if lemma
(\ref{eqLemmaFactorOne}) holds for this determinant, than its value
doesn't contain any term with a numeric factor other than $\pm 1$ and nor
will the product with $\pm s$ do. In the final expansion result there are
no two addends containing symbol $s$ times a product of the same sub-set
of other symbols or with other words there are no addends containing
symbol $s$ and a numeric factor other than $\pm 1$.

As before, $s$ is an arbitrary chosen symbol, hence these considerations
hold true for all symbols.

The subdeterminant is a determinant of dimension $N-1$ and it fulfills the
assumption of the prove, the appearance pattern: The recursion is done,
the same consideration can be done in turn on this subdeterminant, and so
on, till we reach our anchor with dimension $N=2$.

In the next step we need to extend the prove to some more patterns: With
respect to the \emph{symbols} everything has been said but the
coefficients of the LES also have addends $\pm 1$ or with other words
addends with all symbols' power equal to null. Arbitrarily placed ones in
the LES (and hence in the determinant of its sub-matrices) would
immediately break our prove but fortunately the ones also have specific
appearance pattern and similar considerations as for the symbols prove
that lemma (\ref{eqLemmaFactorOne}) still holds.\footnote{The prove could
end here if the implementation of the software would undergo a minor
change: The ones in the LES result from the conditions of specific
devices, mostly the sources. Each of these devices produces ones in pairs,
that appear in the patterns, which are the prerequisite of the prove so
far. If the implementation would handle the ones as device specific
constants, thus make a difference between \emph{this} device's ones and
\emph{that} device's ones then we had already reached our goal. It would
indeed mean a minor implementation change only and no significant perform
loss to do so. However, the actual implementation aggregates all ones in a
single numeric factor and this requires the extension of the prove.}


\subsection{Extension of prove to other devices}

The first part of the prove has been done for matrices with symbols only.
These matrices exhibit the appearance pattern
(\ref{eqPatternsOfAppearance}) for all the symbols and their coefficients
are sums of such symbols and nothing else. We get this kind of matrices
from networks with passive devices only. The prove is now extended to the
matrices, which are seen if the other supported devices are part of the
network.

This part of the prove begins with a single constant voltage source; the
other devices will be similar. The determinants consisting only of sums of
symbols result from passive networks without any sources. If we add a
constant voltage source then the LES is extended by a new row and two new
columns. (One of the new columns is a new right-hand side of the equation
system.) The new unknown is the current through the source and it flows
into one node and returns to the source from another node; this leads to a
column with a vertical pair of $\pm 1$. The new unknown comes along with
the additional equation that says that the difference between the source's
voltage and the potential difference between the connected nodes becomes
null; this leads to a horizontal pair of $\pm 1$ and the one and only one
in the new RHS column. This last mentioned one actually has a negative
sign as the internal representation of the equations is defined such that
summing up all columns (including the ``RHS'') is equal to null. We get
the following structure of the LES. The $S_{ij}$ are the sums of the
symbols:

\begin{equation}
\label{eqLESWithVoltageSource}
\left(
\begin{array}{cccccccc:c}
 S_{11}  &        &  S_{1i}  &        &  S_{1j}  &        &  S_{1z}  & 0       & 0      \\
         & \ddots &          & \ddots &          & \ddots &          & \vdots  & \vdots \\
 S_{m1}  &        &  S_{mi}  &        &  S_{mj}  &        &  S_{mz}  &  \pm 1  &        \\
         & \ddots &          & \ddots &          & \ddots &          & \vdots  &        \\
 S_{n1}  &        &  S_{ni}  &        &  S_{nj}  &        &  S_{nz}  &  \mp 1  &        \\
         & \ddots &          & \ddots &          & \ddots &          & \vdots  & \vdots \\
 S_{z1}  &        &  S_{zi}  &        &  S_{zj}  &        &  S_{zz}  & 0       & 0      \\
0        & \ldots &  \pm 1   & \ldots &  \mp 1   & \ldots &        0 & 0       & -1     \\
\end{array}
\right)
\end{equation}

The upper left area of the matrix consists of coefficients $S$, which are
sums of symbols as seen before, particularly with the typical appearance
patterns for all the symbols. In the implementation the matrix is not
necessarily arranged as shown in (\ref{eqLESWithVoltageSource}), the
additional row and columns may appear at different positions but this has
no impact on the prove since a rearrangement of rows and columns of a
determinant is always possible without another effect then a change of
sign. Which is important is that the pairs $\pm 1$ both appear next to the
symbols, either in vertical or horizontal direction.

All possible sub-matrices, which we have to investigate, will share the
appearance with the matrix. These sub-matrices have the upper-left area of
the matrix besides their last row and column; the last row and column are
taken from the LES somewhere downwards and somewhere to the right,
respectively. Which leads to a determinant with

\begin{itemize}
  \item a symbol area in the upper left corner,
  \item one right-most column with the pattern of one of the two
    right-most columns of the LES and
  \item a bottom row with the pattern of the last row of the matrix.
\end{itemize}

Depending on which sub-matrix is taken any of the three elements might be
not present. Furthermore the pairs $\pm 1$ might be reduced to a single
$\pm 1$ and the right-most column might even contains all nulls.
(actually, even the original matrix pattern may only contain single ones
instead of pairs if one of the two related nodes is the ground node.)

Most of the possible patterns of the sub-matrix do not require a further
consideration. If a column holds only null values it's trivial since the
determinant is null, too. If the new $\pm 1$-rows or columns are not
contained it's the same situation as in the first step of the prove. If a
sub-matrix has inherited the column with the single one then its
determinant can be expanded along this column and -- disregarding the sign
-- it can be replaced by the according only subdeterminant, which doesn't
have such a column.

If a subdeterminant has inherited at least one out of the new row or columns
then we can modify it similar to the first step of the prove. If a pair
$\pm 1$ is present in the last row (column) then one of the columns (rows)
is replaced by the sum of both, which reduces the pair to a single one. We
will always end up with a determinant with modified symbol area and a row
and a column (both optional) with all nulls and a single one. The
modification of such a determinant is show for LES
(\ref{eqLESWithVoltageSource}); be the number of rows $N$, then equation
(\ref{eqLESWithVoltageSourceSubDet}) presents $\left|M^{[N]}_{NN}\right|$
as an example (with no loss of generality the sign of the $\pm 1$ has been
chosen and set):

\begin{equation}
\label{eqLESWithVoltageSourceSubDet}
\left|
\begin{array}{cccccccc}
S_{11}        &        & S_{1i}        &        & S_{1i}+S_{1j}  &        & S_{1z}        & 0      \\
              & \ddots &               & \ddots &                & \ddots &               & \vdots \\
S_{m1}        &        & S_{mi}        &        & S_{mi}+S_{mj}  &        & S_{mz}        & 1      \\
              & \ddots &               & \ddots &                & \ddots &               & \vdots \\
S_{m1}+S_{n1} &        & S_{mi}+S_{ni} &        & S_{mi}+S_{mj}
                                                  +S_{ni}+S_{nj} &        & S_{mz}+S_{nz} & 0      \\
              & \ddots &               & \ddots &                & \ddots &               & \vdots \\
S_{z1}        &        & S_{zi}        &        & S_{zi}+S_{zj}  &        & S_{zz}        & 0      \\
0             & \ldots & 1             & \ldots & 0              & \ldots & 0             & 0      \\
\end{array}
\right|
\end{equation}

The suggested modifications will reduce the subdeterminant always to a row
and column (both optional) with all nulls and a single one and a symbol area.
The expansion is done along the column with the single one and leads to a
single sub-subdeterminant. It has inherited the special row and can be
expanded along this row; this in turn leads to a single
sub-sub-subdeterminant. The latter only contains coefficients from the
(modified) symbol area. We disregard the sign; the absolute value of the
determinant (\ref{eqLESWithVoltageSourceSubDet}) is equal to:

\begin{equation}
\label{eqLESWithVoltageSourceSubDetExpanded}
\left|
\begin{array}{ccccc}
S_{11}        &        & S_{1i}+S_{1j}  &        & S_{1z}        \\
              & \ddots &                & \ddots &               \\
S_{m1}+S_{n1} &        & S_{mi}+S_{mj}
                         +S_{ni}+S_{nj} &        & S_{mz}+S_{nz} \\
              & \ddots &                & \ddots &               \\
S_{z1}        &        & S_{zi}+S_{zj}  &        & S_{zz}        \\
\end{array}
\right|
\end{equation}

$S_{ij}$ denotes the original matrix coefficients in the symbol area. Each
$S_{ij}$ is a sum of any symbols, where the appearance pattern of each
symbol across the $S_{ij}$ in the original matrix is again described by
(\ref{eqPatternsOfAppearance}). To end this step of the prove we still
need to show that in (\ref{eqLESWithVoltageSourceSubDetExpanded}) the
appearance pattern is unharmed for all symbols. If we e.g. look at the
middle column: For all rows but the middle it is the sum of the original
column $j$ with original column $i$. The original row $i$ has been
scratched out during the expansion of the determinant, so the sum of
symbols means that a symbol $s$ from original column $i$ has either been
moved to the middle column or it has been eliminated if in the original
matrix both columns contained this symbol and due to the inverse signs in
the appearance patter. Both possibilities retain the appearance pattern
for $s$. This consideration holds for all symbols in the $S$ and it holds
likewise for the row operation. Its the same situation as explained above
for symbol $t$ in determinant (\ref{eqDetWithSingleSAndTheTs}): The
appearance pattern of the symbols is retained in the relevant
sub-determinant.

Please note: These considerations include the situation, where the symbol
area is empty (meaning the area is null not the coefficients). In which
case the expansion of the subdeterminant will always yield the absolute
value one.

In the first step lemma (\ref{eqLemmaFactorOne}) has already been proven
for symbol determinants, which exhibit the appearance pattern
(\ref{eqPatternsOfAppearance}). In this step we showed that all the
subdeterminants $\left|M^{(k)}_{ij}\right|$ of matrix or LES
(\ref{eqLESWithVoltageSource}) can be rearranged to such a symbol
determinant and consequently at this point the prove is extended to all
the LES with a single constant voltage source.

The considerations of this step can be applied recursively to handle all
the LES with any number of constant voltage sources. Additional sources
add additional rows and pairs of columns to the LES and each have the same
structure with respect to the ones and null values. All of these rows and
columns can be reduced to all nulls and a single one in the same way and
the appearance pattern in the symbol area is still not harmed: If the
required quality is retained in the first iteration than the output of
this iteration is legal input to the same prove or consideration again,
which proves the second iteration, and so on.

The repeated application of the proposed manipulation of the
subdeterminants just needs one additional consideration: The first
iteration will not only affect the symbol area but also the other
additional rows and columns introduced by the other constant voltage
sources. With two sources the LES (\ref{eqLESWithVoltageSource}) could
become:

\begin{equation}
\label{eqLESWith2VoltageSources}
\left(
\begin{array}{ccccccccccc:cc}
S_{11} &        & S_{1i} &        & S_{1j} &        & S_{1k} &        & S_{1z} & 0      & 0      & 0      & 0      \\
       & \ddots &        & \ddots &        & \ddots &        & \ddots &        & \vdots & \vdots & \vdots & \vdots \\
S_{m1} &        & S_{mi} &        & S_{mj} &        & S_{mk} &        & S_{mz} & \pm 1  &        &        &        \\
       & \ddots &        & \ddots &        & \ddots &        & \ddots &        & \vdots &        &        &        \\
S_{n1} &        & S_{ni} &        & S_{nj} &        & S_{nk} &        & S_{nz} & \mp 1  &        &        &        \\
       & \ddots &        & \ddots &        & \ddots &        & \ddots &        & \vdots &        &        &        \\
S_{o1} &        & S_{oi} &        & S_{oj} &        & S_{ok} &        & S_{oz} & 0      & \pm 1  &        &        \\
       & \ddots &        & \ddots &        & \ddots &        & \ddots &        & \vdots & \vdots &        &        \\
S_{p1} &        & S_{pi} &        & S_{pj} &        & S_{pk} &        & S_{pz} &        & \mp 1  &        &        \\
       & \ddots &        & \ddots &        & \ddots &        & \ddots &        & \vdots & \vdots & \vdots & \vdots \\
S_{z1} &        & S_{zi} &        & S_{zj} &        & S_{zk} &        & S_{zz} & 0      & 0      & 0      & 0      \\
0      & \ldots & \pm 1  & \ldots & \mp 1  & \ldots & 0      & \ldots & 0      & 0      & 0      & -1     & 0      \\
0      & \ldots &        & \ldots & \mp 1  & \ldots & \pm 1  & \ldots & 0      & 0      & 0      & 0      & -1     \\
\end{array}
\right)
\end{equation}

Please note, in the example the $\pm 1$ pairs in the two bottom rows share
a column index. This might happen if the sources are connected to the same
node but is neither essential for nor contradictory to our argumentation.
It has been done only to reduce the size of the representation.

Now it is important that the pairs $\pm 1$ in the additional rows and
columns of the LHS of the LES are all in the index range of the symbol
area. Furthermore, the appearance pattern of the ones is one of the
patterns we'd considered for the symbol area. Consequently, the first
iteration will retain this pattern in all the other additional rows
(columns) for the same reasons as explained before for the symbols in the
symbol area. This is why the iteration can then in turn be done on the
next additional row (column).

Because this recursion can be done lemma (\ref{eqLemmaFactorOne}) is now
proven for all the LES created from networks consisting of passive devices
(leading to the symbol area) plus any number of constant voltage sources.

\linnet{} supports more kinds of sources and the ideal operational
amplifier. All of these devices are handled by introducing additional
unknown currents and thus adding a LHS column to the LES and an additional
equation. The constant current source also adds a RHS column. The
character of the additional rows and columns is always the same; the
symbols (for controlled sources only) appear in pairs and the only numbers
are pairs of $\pm 1$ or a single one in a column of nulls otherwise. The
idea of reducing the additional rows and columns to a single value without
destroying the appearance pattern of the symbols in the symbol area is the
same as shown in detail for the constant voltage sources. The determinant
is then expanded along the rows and columns with single ones until the
only remaining subdeterminant only consists of a symbol area which the
statement was proven for in section \ref{secProveOfFacOne}. In general, we
will never see a numeric factor other than $\pm 1$ in the coefficients of
the LES.


\subsection{Intermediate expressions}

An important note for the implementation is that the prove only holds for
coefficients that are the result of a completed iteration of the
elimination. There's no such prove for intermediate results during the
elimination step. Actually, the elementary step is implemented by first
adding two products of two coefficients each and then doing the
factorization. The intermediate sum can indeed have factors other than
$\pm 1$ due to identical addends in both products. This is one of the
reasons why the implementation still uses integers instead of Booleans as
numeric factors.

Nonetheless, the prove is even useful for the intermediate results: An
implication of the prove is that the numeric factor appearing in the
intermediate result can only grow incrementally, i.e. one by one with the
result's addends being figured out in a loop. (In far the most cases it
still stays absolute one.) The implementation foregos a range check for
the required integer operations: It's proven that an overrun can occur
only after a pseudo-infinite computation time and an error report which
would be printed in an unreachable future is useless and doesn't add any
value to the software or its behavior.


\section{User-defined voltages}

It can be proven that the properties of the coefficients -- no symbol
powers other than null or one and all numeric factors absolute one -- hold
for the user-defined voltages, too. User-defined voltages are differences
of two node potentials, where the two nodes are arbitrarily chosen. If one
of the nodes is the ground node then the difference is trivial, either an
existing node potential or the negated value. Obviously the properties are
retained in these trivial results. It's less evident for the general
situation of two nodes, whose potentials are both unknowns of the solved
LES. The symbol powers are still simple, as we compute a difference
there's no way to get new powers in the products of symbols. However there
could be addends with same symbol products and same sign of the numeric
factor, which would lead to a numeric factor of two but no longer absolute
one. This can be excluded by looking at the LES and how it is transformed
by using derived unknowns.

Be $M$ the matrix, which represents a LES with m unknowns and $n-m$
knowns:

\begin{equation*}
M =
\left(
\begin{array}{ccc:ccc}
a_{11} & \ldots & a_{1m} & a_{1, m+1} & \ldots & a_{1n} \\
\vdots &        & \vdots & \vdots     &        & \vdots \\
a_{m1} & \ldots & a_{mm} & a_{m, m+1} & \ldots & a_{mn}
\end{array}
\right)
\end{equation*}

Be the user-defined voltage $\hat U$ the difference of the unknowns in column $u$
and $v$, i.e. $\hat U = U_u - U_v$. This equation could be put into the LES in
order to directly figure out the user wanted $\hat U$. We substitute e.g.
$U_u$ by $U_u = \hat U + U_v$. The LES is transformed into an equivalent
LES using a new vector of unknowns, where $\hat U$ replaces $U_u$:

\begin{equation*}
\hat M =
\left(
\begin{array}{ccccccc:ccc}
a_{11} & \ldots & a_{1u} & \ldots & a_{1v} + a_{1u} & \ldots & a_{1m} & a_{1, m+1} & \ldots & a_{1n} \\
\vdots &        & \vdots &        & \vdots          &        & \vdots & \vdots     &        & \vdots \\
a_{m1} & \ldots & a_{mu} & \ldots & a_{mv} + a_{mu} & \ldots & a_{mm} & a_{m, m+1} & \ldots & a_{mn}
\end{array}
\right)
\end{equation*}

\noindent
The new unknown $\hat U$ inherits the coefficients of the
substituted unknown $U_u$ and the coefficients of unknown $U_v$ become the
sum of two columns of the original matrix $M$. To apply the elimination
method according to section (\ref{secEliminationMethod}) for $\hat U$ we
need to swap two columns so that $\hat U$ becomes the last unknown in the
again new vector of unknowns:

\begin{equation*}
\tilde M =
\left(
\begin{array}{ccccccc:ccc}
a_{11} & \ldots & a_{1m} & \ldots & a_{1v} + a_{1u} & \ldots & a_{1u} & a_{1, m+1} & \ldots & a_{1n} \\
\vdots &        & \vdots &        & \vdots          &        & \vdots & \vdots     &        & \vdots \\
a_{m1} & \ldots & a_{mm} & \ldots & a_{mv} + a_{mu} & \ldots & a_{mu} & a_{m, m+1} & \ldots & a_{mn}
\end{array}
\right)
\end{equation*}

The solution for $\hat U$ is an expression consisting of the $n-m+1$
right-most coefficients of the last row of $\tilde M$ after the last
elimination step. Using the position indexes of $\tilde M$ these are
$a^{(m-1)}_{mm}$ till $a^{(m-1)}_{m,n}$. According to lemma
(\ref{eqLemmaCoefIsDet}) these coefficients are the determinants of
$\tilde M^{[m]}_{mm}$ till $\tilde M^{[m]}_{mn}$, respectively.

Disregarding the sign change due to the swapping of columns,
$|M^{[m]}_{mm}|$ is the unchanged system determinant: If we replace column
$v$ with the difference of itself and column $m$, which doesn't alter the
determinant's value according to lemma (\ref{eqLemmaLinCombRowsD}), then
we have the same determinant as for the original LES $M$.

The $n-m$ other determinants share the left-most $m-1$ columns of $\tilde
M$ and have columns $m+1$ till $n$ of $\tilde M$ as right-most column.
These determinants contain the column $v$ with the sum of two original
coefficients but don't contain the column where the added coefficients
origin (i.e. column $m$ of $\tilde M$ or column $u$ of the original matrix
$M$). With respect to the appearance patterns of symbols according to
(\ref{eqPatternsOfAppearance}) this means a move of the symbols from the
original column $u$ to column $v$, which doesn't harm the patterns. Due to
elimination particular symbols might show a different pattern but all
appearing patterns are still as listed in (\ref{eqPatternsOfAppearance}).
Since the appearance patterns are the prerequisite for the proves of the
properties of the coefficients of the elimination steps these proves still
hold for the $n-m$ coefficients which form the numerators of the solution
for $\hat U$.

All coefficients that form the solution of the user-defined voltage $\hat
U$ have the same properties as shown for the coefficients of the actually
solved LES with the original unknowns. Also in the user-defined voltages
we will never see a numeric factor other than absolute one and there won't
be any symbol with a power of higher than one.

\section{Implementation of elementary step}

This section explains the implementation of the elementary step of the
elimination method (see section \ref{secEliminationMethod}). This
operation combines the four matrix coefficients under progress and the
common divisor to a new coefficient, the resulting coefficient of that
elimination step. The implementation is not done very explicit but uses
simplifications and omissions, which are justified in the following. The
source code is not self-explaining and will be hard to understand without
the explanations given here.

The elementary step is defined as

\begin{equation}
\label{eqElemStep}
a^{(k)}_{ij} = \frac{a^{(k-1)}_{kk} a^{(k-1)}_{ij} - a^{(k-1)}_{ik} a^{(k-1)}_{kj}}
                    {a^{(k-1)}_{(k-1)(k-1)}}
\end{equation}

\noindent
in elimination step $k$. For all affected $a^{(k)}_{ij}$ -- including the
resulting $a^{(k)}$ at the LHS of equation (\ref{eqElemStep}) -- it holds
due to former proves:
\begin{itemize}
  \item $a$ is a sum of addends. Each addend is a product of symbols to
    the power of either null or one and a numeric factor of either plus or
    minus one
  \item No two addends of $a$ have the same product of symbols, i.e. the
    vectors of powers are different between all pairs of addends
\end{itemize}

It follows that the result of the division can be carried out without a
remainder. With other words, the numerator of equation (\ref{eqElemStep})
step can be brought into the shape of a product of a coefficient with the
above properties and the denominator of equation (\ref{eqElemStep}).

The implementation basically follows the definition of the elementary
step. It carries out the first product of the numerator by multiplying and
summing up all pairs of addends between first and second factor. This is
followed by the second product; here the products of the pairs of addends
are immediately subtracted from the sum so far. After this step a
polynomial division is carried out.

The mentioned omissions relate to such (intermediate) terms we can predict
for that they won't appear in the final result because of the known
properties of this result. These terms are simply discarded but neither
summed up nor stored.

A hypothetic implementation of the data representation is introduced,
which can represent all coefficients and all intermediate and the final
result (referred to as ``variable'' in the following). This implementation
is actually not used but below we will justify why it can be safely
reduced to the actually implemented variable representation.

The product of the addends of the coefficients in the numerator can
produce symbol powers of 0, 1 or 2. The hypothetic implementation of a
variable will use a 2 Bit integer value to store a symbol's power. It'll
furthermore order these two Bit values in a certain order of symbols,
which is defined arbitrarily but fixed for all variables. (Alphabetic
order of symbols would e.g. suffice.) Reading this sequence of bits as a
whole as an unsigned binary number gives us a one-by-one relation between
any appearing combination of symbols and an integer number. We call this
number the ``value'' of the symbols or simply symbol value of an addend of
the variable. In this sense products of symbols become an ordered set.

The addends of a variable will always be ordered in decreasing order of
their symbol value.

For the input variables we know that the numeric factor of each addend is
$\pm 1$ but this is not proven (and indeed not the truth) for the
numerator in equation (\ref{eqElemStep}); its two addends can contribute
with same symbol combinations. Therefore the implementation needs to use
an integer type here.

For the further we assume that all input variables, all intermediate
results and the final result are represented in the outlined form.

The order of addends according to their symbol value is important for the
further and therefore we now have a look at the required operations product
and quotient of addends with respect to the symbol value:

The product of two addends is figured out by multiplying their numeric
factors. The result is an integer as the operands are. The products of
symbols are multiplied by adding the powers. All the 2 Bit fields have to
be added separately, one pair of 2 Bit values for each symbol. However,
since we only add two values each either null or one, we will never see an
overrun in any of the 2 Bit sums and therefore the result of all the 2 Bit
sums is identical to the arithmetic sum of the symbol values of the
addends. This implies for example that the product of a variable with an
addend will not change the order of addends -- the symbol value of all
addends rises by the same value, the symbol value of the multiplied
addend.

The quotient is analog. Now, the symbol powers are subtracted. Instead of
subtracting all the 2 Bit values of the distinct symbols separately we can
directly subtract the entire symbol values of the affected addends. We
will only perform divisions which are possible; this is why there won't
ever be an underrun in any of the 2 Bit differences and this is why the 2
Bit operations again can be computed as a single arithmetic difference of
symbol values.

Based on what has been said so far and adding some straightforward
algorithmics the numerator of equation (\ref{eqElemStep}) can be figured
out and stored in the hypothetic data representation. The products of
coefficients are reduced to the sum of all crosswise products of addends
and the sum of addends is a matter of list sorting according to the symbol
value of the addends and adding the numeric factors.

In the hypothetic implementation the polynomial division starts when all
addends of the numerator in equation (\ref{eqElemStep}) are figured out. A
loop iterates along the numerator variable in decreasing order of the
symbol value, i.e. from the left to the right. The computation finishes
when the last addend of the numerator has been processed. The first
operation is to divide the addend under progress by the first, left-most
addend of the denominator variable. Three possibilities exist:

\begin{compactenum}[a)]
  \item The division is possible with respect to the symbols (i.e. the
    numerator addend contains a superset of symbols of the denominator
    addend) and results in symbol powers of either null or one
  \item Like a) but at least one symbol power of two results
  \item The division is not possible (i.e. the numerator addend lacks
    at least one symbol from the denominator addend's set)
\end{compactenum}

Please note, that the division of the numeric factors of the addends will
always succeed without a remainder: The denominator variable is an
unmodified input coefficient, which is proven to have numeric factors
of $\pm 1$ only.

The hypothetic implementation continues in all three cases:
\begin{compactenum}[a)]
  \item The result of the division is put into the result variable as
    its next addend
  \item Like a) but here we can predict that a later contribution to
    the result will eliminate this addend again; we know that the final
    result won't contain addends with symbol powers greater than one
  \item The numerator addend is left as is in the numerator variable. For
    now this is a remainder of the polynomial division. The operation is
    completed and the loop of the polynomial division will handle the next
    numerator addend
\end{compactenum}

In cases a) and b) the back-multiplication is carried out as a principle
of the polynomial division. The new result addend is multiplied with all
addends of the denominator and each product of addends is subtracted from
the numerator variable, which gets the meaning of the still undivided
remainder of the operation. The number of products is finite and defined
by the given length of the denominator variable. Obviously, the first
subtracted product eliminates the addend under progress of the numerator
variable, the following are sorted into this variable; they might
eliminate existing terms, constitute new ones or change the numeric factor
of existing ones.

All variables are ordered in falling symbol value of the addends. This
means that the first subtracted product, which is got by
back-multiplication with the first addend of the denominator has a higher
symbol value than all other addends of the back-multiplication. This first
addend eliminates the addend under progress in the numerator; all further
might remain in the numerator. Due to the order of addends of the
numerator the addend under progress is the one with the currently still
highest symbol value. Consequently and although the back-multiplication
can lengthen the numerator variable it can be said that the highest
remaining symbol power in the numerator is lower than before (remainders
according to case c) are disregarded here) and that all possibly new
addends in the numerator are to the right of the addend under progress and
will be processed later in the main loop.


\subsection{Finiteness of algorithm}

The last statement already proves the finiteness of the algorithm.
Although the numerator can become longer due to the back-multiplication
the highest symbol value, i.e. the symbol value of the addend under
progress, is lowered in every cycle of the main loop. At latest when this
symbol value is lower than the symbol value of the left-most addend of the
denominator the algorithm would terminate: If the symbol value of the
divisor's addend is greater than the symbol value of the numerator's
addend then at least one symbol is missing in the numerator's symbol
product to make the division possible. The addend under progress and all
further addends (which have even lower symbol values) would belong to case
c) and hence be part of the remainder of the division.


\subsection{Omission of addends}

The real implementation of the elementary step discards some addends during
computation of the numerator of (\ref{eqElemStep}) and during
back-multiplication in the polynomial division. When reading the source
code this looks arbitrarily if not faulty. The prove why it can be
safely done follows.

Each addend is tested whether the division with the left-most addend of the
denominator is possible. As the numeric factor of all the denominator
addends is absolute one this is just a matter of ``fitting'' symbols: All 2
Bit power values of the tested addend need to be greater or equal to the
according 2 Bit values of the denominator addend.

Let's have a closer look at case c) of the hypothetic implementation: The
addend under progress can't be divided by the denominator's addend and it
is preliminarily left as an addend of the remainder of the quotient. From
our prerequisites we know that the division can be carried out without a
remainder so at the end of the operation the left addend must have
disappeared. On the first glance this seems to be possible by elimination
of the unwanted addend by subtracted addends from the back-multiplication
with later computed result terms. However due to the sort order according
to the symbol values we can exclude this: All subsequently handled numerator
terms have lower symbol values and consequently no subsequently subtracted
addend from the back-multiplication will ever have a symbol value greater
or equal to the symbol value of the unwanted addend and it would remain till
the end of the complete operation. This is contradictory to the prove of a
remainder-less quotient. Which means that it can't ever happen that the
main loop encounters a numerator addend under progress, which can't be
divided by the left-most denominator addend. Case c) was necessary for the
explanation but doesn't ever occur and needs not to be considered by the
actual implementation.

The last statement does not mean that there can't be any addends in the
numerator, which can't be divided by the left-most denominator addend --
it's rather a matter of \emph{when} they may appear: They can appear
during and at the end of the computation of the sum of products in the
numerator of (\ref{eqElemStep}) and they can even remain in the numerator
variable during the polynomial division until the main loop processes
their preceding addend -- at latest now they will be eliminated by addends
from the back-multiplication. Earlier elimination is also possible: During
computation of the numerator expression such addends might first appear as
a result of the first product and later be eliminated by addends from the
second product. Anyhow, it's for sure that they will be eliminated from
the numerator variable prior to the attempt to divide them by the
left-most addend of the denominator. Instead of keeping track of addends
that will be anyway eliminated somewhere in the future the real
implementation decides to discard them immediately. If one of the products
yields an indivisible addend it is thrown away without any ado. Most
likely, counterparts of these addends will appear during the
back-multiplication; now the implementation needs to throw them away, too
-- elimination in the numerator by subtraction can't succeed due to the
former decision.

Similar considerations hold for case b). The products in the numerator of
(\ref{eqElemStep}) can yield symbols to the power of two. If such an
addend is divisible by the left-most addend of the denominator we can get
a result term with symbol powers of two. From our prerequisites it's yet
for sure that this addend won't appear in the final result. So there must
be a later contribution to the result, which eliminates the unwanted
result term again. However, due to the sort order of the numerator
addends, all later result addends will have symbol values that are lower
than the one of the unwanted addend and the elimination of an already
found result term is simply impossible. This is a contradiction with the
prove of symbol powers only being null or one. Which means that it can't
ever happen that the main loop encounters a numerator addend under
progress, which yields a power of two for at least one symbol when being
divided by the left-most denominator addend.

The rest is just the same: Such addends may temporarily occur in the
numerator variable but they are surely eliminated before the main loop
reached them, before they become the addend under progress. Elimination
can occur due to negated addends during the computation of the second
numerator product or due to subtraction of back-multiplication addends in
earlier cycles of the main loop of the polynomial division. Anyhow, the
main loop will never encounter and process such an addend and this is why
the real implementation immediately discards such addends during the
computation of the numerator of (\ref{eqElemStep}) and why it then has to
throw according counterpart-addends away during back-multiplication.

A further optimization becomes possible in the real implementation: Any
numerator addend which yields a power of two for at least one symbol after
division is immediately thrown away, thus not stored in the numerator
variable. All stored addends yield symbol powers of either null or one
after division by the left-most addend of the denominator. The real
implementation doesn't safe the addends of the numerator of
(\ref{eqElemStep}) but the addends of (\ref{eqElemStep}) \emph{after}
division by the left-most addend of the denominator. This is possible
because any addend which can't be divided for one of the two criteria is
discarded, as said. Consequently, the real implementation can use 1 Bit
fields for the symbols.

The hypothetic implementation reduces the product and quotient of symbol
powers to the arithmetic sum and difference, respectively. The real
implementation uses bit operations. The operations are equivalent since
the test ``division is possible'' ensures that there is neither an over- nor
an underrun of symbol powers from one field into another one. The same
holds for the two tests themselves: Since the power of three doesn't occur
anywhere the hypothetic arithmetic 2 Bit operations can be carried out by
a combination of bit operations only. All of these optimizations lead to
an efficient real implementation, which is equivalent to the hypothetic
implementation used in the considerations here.

A last remark is required. The real implementation doesn't store the
(relevant) addends of the numerator of (\ref{eqElemStep}) but the same,
already divided by the left-most addend of the denominator. This does not
change the argumentation of the hypothetic implementation: The division
means that the symbol value of the left-most addend of the denominator is
subtracted from all addends in the numerator variable. The order of
addends is not effected and nor are the considerations affected, which
base on the statement that a symbol value as large as the symbol value of
the addend under progress will never be seen again in the
back-multiplication of this cycle or in any addend of any later cycle of the
main loop of the polynomial division. All appearing symbol values are lowered
by a constant value but their mutual relations are not changed.


\subsection{Numeric factor one}

In section \ref{secProveOfFacOne} the prove was given that the numeric
factor in the addends of all appearing coefficients will only be $\pm 1$.
The prove given here refers to this statement only at one point: The
division of the numeric factor of the addend under progress by the factor
of the first addend of the denominator would always succeed as the latter
is absolute one. Actually, this just shortens the explanations above but
is irrelevant for the correct operation of the actual implementation. With
one minor exception the implementation of the elementary step (and the
elimination method according to section \ref{secEliminationMethod} anyway)
would continue to work with LESs, where the addends can have arbitrary numeric
integer factors (but still symbols, which are restricted to a power of
null or one).

The critical point is the decision a), b) or c). With arbitrary integer
factors we could expect addends, which are indivisible with regard to a
remainder-less result of the quotient of the numeric factors. But
actually, this can never happen for the same reason why case c) couldn't
occur above: A remainder in the quotient of the numeric factors would mean
to leave the addend under progress with reduced numeric factor in the
numerator variable. Due to the consideration with strictly monotonically
decreasing symbol values this addend could not be eliminated any more - a
contradiction to the prove of a remainder-less final result.

A difference to the same consideration for the quotient of products of
symbols is that the addends of the numerator computation, which show
indivisible numeric factors must not be discarded. Here, combination with
later addends (i.e. either in the further computation of the numerator of
equation (\ref{eqElemStep}) or in a later back-multiplication of the
polynomial division) can change them to hold well divisible numeric
factors but won't surely eliminate them as shown for addends with
indivisible symbol products.

The mentioned minor exception from the irrelevance of the factor-one
statement is the integer overflow handling. If the implementation would be
applied to a system with arbitrary integer factors overflow recognition
becomes a must.\footnote{Actually, there's one more change required on the
implementation: The $\pm 1$ statement is double-checked by assertions.
These assertions needed to be replaced by an assertion that proved the
remainder-less division of the numeric factors.} The current use case may
however safely ignore integer overflows due to the restriction of the
factors to $\pm 1$.