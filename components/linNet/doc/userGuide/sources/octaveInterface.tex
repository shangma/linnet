\chapter{Understanding the Octave Interface}
\label{secOctaveInterface}

\section{Folders, files and functions}

If the switch \code{-o} is put on the command line of \linnet{} then the
computed formulas are exported to Octave in form of a set of generated
scripts (files \file{*.m}). Octave needs to be installed in order to make
any use of these files. A further prerequisite is the availability of
package control, which should normally be installed with
Octave.\footnote{Type \code{pkg list} in the Octave command line to find
out.} Please visit http://www.gnu.org/software/octave/ for information and
download.

One dedicated script is generated for each of the results specified in the
netlist file. The script has the name of the result in the netlist file.
If no result is specified in the netlist file then \linnet{} uses the
implicit result definition \ident{allDependents} and the generated Octave
script has this name, too. All of the result related scripts are bundled
in a common folder, which is also created and which has the name of the
netlist file (except for the removed name extension). Several such folders
will be created if the \linnet{} command line specifies several netlist
files. All of the created folders are placed in a common target directory,
which can either be specified as argument of command line option \code{-o}
or defaults to the current working directory. This directory needs to
exist, \linnet{} won't create it.

The generated, result related Octave scripts implement Octave functions.
The functions in turn have the name of the result in the netlist file.
They are self-contained in the sense that they can be called just like
that if only the current working directory in the Octave session is the
folder containing the script file, i.e. the folder that has the name of
the processed netlist file. If you \code{cd} to the created folder and
type the name of the result then you should get an Octave figure window
with either a step response or a Bode plot -- which one depends on the way
the result was defined. The details of the result definition can be found
at \ref{secResultSpec} on page \pageref{secResultSpec}.

The next step should be to type \code{help} followed by the name of the
result.

The generated result related scripts use some (simple) sub-routines. These
subroutines are implemented as a set of Octave files, which are bundled in
folder \file{private} that is found inside the created netlist related
result folder. The contents of folder \file{private} are invariant. As a
matter of fact, the whole folder is just the copy of a template folder
located somewhere in the installation of \linnet{}. Therefore, getting
this folder in every netlist related result folder is actually not
essential and may even become annoying. (The design decision to do so has
been taken for convenience, to make the result scripts self-contained and
immediately usable.) To avoid these redundant files you can consider to
make a copy of the files in the private folder into any folder holding
Octave script files and to add this other folder to the Octave search
path. From now on you would run \linnet{} with command line option
\code{-i} and folder \file{private} is no longer generated in the created
result folders. The copied files on the Octave search path will be used
instead of those in folder \file{private}.


\section{The two kinds of generated Octave functions}

Depending on how a user-defined result is specified in the netlist file
the generated Octave script will define a function of one out of two
possible kinds. If the netlist command \code{PLOT} is used then the
generated Octave function will show the Bode plot of a transfer function (or
frequency response) of the system or one of its variables. If the netlist
command \code{RES} is used then the generated Octave function will show
the step response of the system or one of its variables.

The plots are presented if the function is called without assignment of
the (optional) return values. The simplest way to do so is to just type
the function name (i.e. result name) on the Octave command line.


\section{Symmetries between the two kinds of generated Octave functions}

Regardless of the fundamentally different plots the two kinds of generated
Octave functions are quite similar.

Both kinds of functions have (nearly) the same signature. The difference
is only their internal behavior, the internal decision to chose an
appropriate kind of plot and this behavior is inhibited if a return value
of the function is consumed by the calling code. If so, both kinds of
functions behave identical: They generate and return the LTI object and
maybe the parameter set and a default vector of either frequency values or
time designations. (This last, optional return value is the only remaining
difference in the behavior of the two kinds of functions.) As the first
two return values are far the most relevant ones, both kinds of functions
can mostly be used in an identical fashion. This means for example, that a
Bode plot result can easily be used to plot a step response and vice
versa:

Be \file{G.m} an Octave script generated from a \code{PLOT} command and
\file{I\_Uin.m} an Octave script generated from a \code{RES} command then
will the direct call of
\begin{itemize}
  \item \code{G} open a figure window with the Bode plot of (transfer
    function) $G$. This is the internal behavior of \code{PLOT} generated
    M functions,
  \item \code{I\_Uin} open a figure window with the step response of
    (current) $I_{U_{in}}$. This is the internal behavior of \code{RES}
    generated M functions.
\end{itemize}
In contrast to this will a the first return value consuming call like
\begin{itemize}
  \item \code{step(G)} open a figure window with the step response of the
    system described by (transfer
    function) $G$,
  \item \code{bode(I\_Uin)} open a figure window with the Bode plot of the
    transfer function from system input $U_{in}$ to current $I_{U_{in}}$.
\end{itemize}
Please note, that this is an example. In general, there are constraints.
As an example, Octave refuses the Bode plot for systems other than SISO.

Using the generated M functions as argument of enclosing Octave
expressions works well but is somewhat slow: Each time the
expression is evaluated another (temporary) LTI object is created by the
function. This causes a noticeable delay. In most situations it will be
more appropriate to create the LTI object once in the workspace and to
continue with the same object. The commands from the last example would
change to:
\begin{verbatim}
  tfG = G; step(tfG)
  tfI_Uin = I_Uin; bode(tfI_Uin)
  ...
  nyquist(tfG)
\end{verbatim}
The last line is just meant to demonstrate how to benefit from the once
and only once created objects.


\section{Varying device values}

A netlist file, which is the source of the Octave scripts that create the
LTI objects, has the option to specify device values. This applies to the
passive devices and the controlled sources. A resistor can get a value,
which is understood as resistance in $\Omega$, a conductance can get a
value in $\frac{1}{\Omega}$, the capacitor is understood in $F$ and the
impedance in $H$. The device values of controlled sources are understood
as the proportionality factor between control voltage/current and controlled
voltage/current. Accordingly, they are dimensionless or understood as
either $\frac{A}{V}$ or $\frac{V}{A}$.

If the netlist specifies a device value then this value is put into the
generated Octave script as default value for all numeric computations. For
\linnet{} itself the value is meaningless, it is not used at all by any
\linnet{} computation.

All devices which no value is specified for in the netlist will get a
hard coded standard value as default value in the generated Octave script.
These standard values depend on the kind of device but do not consider
values, which might have been specified in the netlist for other devices. A
mixture of explicitly specified device values and standard values in one
and the same netlist file is not recommended.

\begin{table}[bt]
\begin{center}
\begin{tabular}{|l|c|c|l|}

\hline

% The column headers:
Device & Symbol & Value & Remark \\
\hline
% The next hline makes the column titles a separate rectangular box,
% comment it out to get just a line as separator.
%\hline

\hline

% Table entries start here.
Resistor                          & R    & 100 $\Omega$     & RLC should yield audio frequencies \\
Conductance                       & Y    & 10 mS            &                                    \\
Capacitor                         & C    & 10 $\mu F$       &                                    \\
Impedance                         & L    & 1 mH             &                                    \\
Voltage controlled voltage source & U(U) & 1                & No particular use case aimed       \\
Current controlled voltage source & U(I) & 1 $\frac{V}{A}$  & No particular use case aimed       \\
Voltage controlled current source & I(U) & 5 $\frac{mA}{V}$ & Use case unipolar transistor       \\
Current controlled current source & I(I) & 250              & Use case bipolar transistor        \\

\hline

\end{tabular}
\caption{Device standard values}
\label{tabDeviceStdValues}
\end{center}
\end{table}

The hard coded standard values are chosen such that if no device values are
specified in the netlist then the time constants of the system will
probably be in the audio frequency range. This holds for the values of
the four passive device kinds.

The standard values for the controlled sources are more difficult to
choose. Such sources can be used to model totally different physical
effects and a prototypic value is hard to find. If there is a typical use
case like the bipolar transistor simulation with a current controlled
current source then this use-case has let to the chosen standard value.
Because of the difficult definition of these standard values it is
recommended not to use controlled sources in numerical evaluations with
Octave without the explicit specification of a realistic device value in
the netlist.

All hard coded standard values of all affected devices are listed in table
\ref{tabDeviceStdValues}.

The values -- either explicitly specified or standard values - are written
into the generated Octave script and used if the scripts are run without
more ado. In the Octave session they can be accessed and changed. This way the
numeric system simulation can be repeated with altered values and the
system behavior can be investigated in dependency of device value changes
in an empiric way. 

The second return value of the generated Octave functions is a
\code{struct}. The members have the names of the devices. Each member holds
the value of the related device. The struct implements the parameter set
the returned LTI object has been created from.

The third return value of the generated Octave functions depends on the
kind of result, Bode plot or full result. It is a vector of frequency
values or time designations, respectively. Frequency values have the unit
$\frac{rad}{s}$ as common for Octave functions. The retrieved vector is a
suitable starting value for system simulations and can be used with the
Octave functions \ident{bode}, \ident{step}, \ident{impulse}, etc. It
depends on the kind of command, whether a frequency or time vector is
required.

Calling the generated function with consumption of at least one return
value inhibits the internal behavior; no figure window is opened, no plot
is made. Just the mentioned data is fetched if the return values are
assigned to workspace variables. These variables can then be altered and
reused for subsequent computations.

The generated functions take a parameter set as only (optional) argument.
In most cases, the parameter set will be the modified return value of a
previous call of the function. Providing this argument makes the functions
behave exactly as before but only using the altered parameters. You can
either let them plot the standard figures (Bode plot or step response) or
let them create a \emph{new} LTI system object based on the altered
parameters. Please note, it is not possible to change the parameters of a
previously created, already in the workspace existing LTI object; instead,
you will always have to replace or overwrite it.

Varying the device constants or parameters of a circuit is now completed
with an example. Be \file{G.m} the generated Octave script:

% The flushleft (table left-aligned) in conjunction with @{} (cell
% contents without leading blanks space) ensures left-aligned appearance
% of the M code.
%   The \hspace* at the end of the longest M code cell enlarges the
% distance of code and explanations.
\begin{flushleft}
\begin{tabular}{@{}ll}
\verb+G+             & \% Plot the transfer or step function as defined in the circuit file \\
\verb+[tf_G p] = G+  & \% Get an LTI object (class tf) and the default set of device constants \\
\verb+p.R1 = 330;+   & \% Modify device constant $R_1$ in parameter set \ident{p} \\
\verb+p.R2 = 2*p.R1+
\hspace*{8mm}        & \% Modify device constant $R_2$ in parameter set \ident{p} \\
\verb+G(p)+          & \% Repeat the initial plot with the altered parameter set \ident{p} \\
\verb+tf_G = G(p)+   & \% Replace the LTI object with a new one using the altered device constants\\
\verb+step(tf_G)+    & \% Plot the step response of the system with modified device constants \\
\verb+bode(tf_G)+    & \% Plot the transfer function with modified device constants \\
\verb+nyquist(tf_G)+ & \% Nyquist plot with modified device constants \\
%\verb++ & \%  \\

\end{tabular}
\end{flushleft}


\section{Caveats of Octave}

Although generally working well, has using Octave for numerical
post-processing some weaknesses. The author has solely used Windows
ports of Octave which might be worse than Linux releases with respect to
functional failures and stability. The following is stated only for the
Windows ports of release 3.4.3 and 3.6.4 running under Windows 7.

Octave 3.6.4 has general problems with the figure windows. The chosen
graphics back-end was ``fltk''.
\begin{itemize}
  \item The opened figure windows are initially black. The actual plots
    appear only after some clicking into the window pane or on the
    buttons to enforce a redraw    
    
  \item If the figure window is enlarged by dragging the corners with the
    mouse than only the frame of the window is resized, the contents stay
    the same. This is particular annoying when MIMO systems produce grids
    of plots; these are initially tiny but can't be enlarged
\end{itemize}

These problems let to the decision to continue with the elder release
3.4.3. Window handling is okay in this release.

With Octave 3.4.3 we get a reproducible crash when running result \code{G}
of netlist file
\file{bandStop.makesOctave3-4-3-win32Crash.cnl}.\footnote{The file is
contained in the source distribution of \linnet{}.} The \linnet{}
generated file \file{G.m} looks however unsuspiciously and resembles
dozens of other generated files which are successfully processed. Octave
3.6.4 and MATLAB process file \file{G.m} successfully and plot the
expected result.
