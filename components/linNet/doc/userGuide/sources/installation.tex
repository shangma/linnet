\chapter{Installation}

A valid installation looks like this: The \linnet{} application is a
simple executable file plus a folder with resources for the Octave code
generator; this folder is called \file{private} and is located next to the
\linnet{} executable file, i.e. both are siblings in the same parent
directory. The environment variable \code{LINNET\_HOME} points to this
parent directory.

Accordingly, this is what you have to do to install \linnet{}: 

Create an empty folder, the installation directory, e.g.
\file{/ProgramFiles/linNet}. Copy the \linnet{} executable file into this
directory. Copy the folder \file{private} into this folder. Create a
persistent environment variable \code{LINNET\_HOME} and assign the path to
the installation directory, \code{/ProgramFiles/linNet} in our example.

The usual way to use \linnet{} is to run it from a shell window. You will
probably organize your input circuits in a folder or a set of such and
\code{cd} in your shell to this folder. To run \linnet{} in this scenario
in a convenient way it is useful to place a reference to the installation
directory in the system environment variable \code{PATH}. In our example
you'd append \code{:/ProgramFiles/linNet} (UNIX) and
\code{;\bs{}ProgramFiles\bs{}linNet} (Windows) to the current value of
this variable.

The definition of a shell command alias pointing to the \linnet{}
executable file can be a reasonable alternative to extending the
environment variable \code{PATH}.

If the environment variable is not set then \linnet{} makes an attempt to
find its resources via the path to itself. However, it depends on the
system, the shell in use and the way the command is typed, whether this
information is completely passed to the application or not. If it can't
find out where to find itself it'll refuse to generate Octave code as it
lacks the folder \file{private} with code templates.\footnote{There are
system specific functions which let an application find out, where its
executable file is located, e.g. \code{GetModuleFileName} for Windows, but
the use of such functions would make \linnet{} system dependent. This
price is much too high for an a bit simpler installation.} Not setting the
environment variable can be a safe option if you decide to run \linnet{}
solely from a shell window with an alias, which expands to the full,
absolute path of the executable.


\section{Mac OS}

Unfortunately, Macintosh users will have to compile \linnet{} from the
sources prior to use. At the moment no pre-built executables are available
for this system. Please refer to section \ref{secBuildingLinNet} on page
\pageref{secBuildingLinNet}.


\section{Linux}

The download of \linnet{} contains pre-build executable files for Linux
(in DEBUG and PRODUCTION compilation). The executable files are already
bundled with the resources folder \file{private} so that the installation
reduces to copying the build folder of choice to any location, preferably
\file{/usr/bin}, renaming the folder itself to \file{\linnet{}}, running
\code{chmod +x} to make the binary file executable and setting the
environment variable \code{LINNET\_HOME}; it would be set to
\file{/usr/\-bin/\-\linnet{}} if the suggested installation folder was
used. The chosen installation path should not contain blanks.

The Linux files have been built and run in a single environment only; this
was a Fedora 18 distribution. It can't be guaranteed that the binaries are
compatible with any other Linux derivate. Furthermore, no profound testing
has been done under Linux, only the few test cases from the distribution
have been run to prove that the expected program output is yielded. No
testing with Octave has been done under Linux.


\section{Windows}

\subsection{Pre-built executables}

The download of \linnet{} contains pre-build executable files for Windows.
A build for 32 and 64 Bit is contained, each as DEBUG or PRODUCTION
compilation. The executable files are already bundled with the resources
folder \file{private} so that the installation reduces to copying the
build folder of choice to any location, preferably
\file{c:\bs{}\-Program\-Files}, renaming the folder itself to
\file{\linnet{}} and setting the environment variable \code{LINNET\_HOME};
it would be set to \file{c:\bs{}\-Program\-Files\bs{}\-\linnet{}} if the
suggested installation folder was used. The chosen installation path
should not contain blanks.


\subsection{Required DLLs}

The build of \linnet{} has the executable file \file{linNet.exe} as only
product. This executable file is however not self-contained; the
application is linked against the compiler's C libraries, which in case of
the GNU compiler collection build on some Windows DLLs. These DLLs need to
be available in order to execute \linnet{}. Normally, this should always
be the case. In the rare case that such a library should be missing on
your system then the only way out will be to install the GCC port of MinGW
and to rebuild \linnet{}. If you can run the simplest \code{helloWorld}
application compiled with GCC then you should also be able to run
\linnet{}.

For a Windows 7 system and using the 32 Bit build the following list of
DLLs appeared to be loaded by \linnet{}. For different Windows releases or
for the 64 Bit build the list may vary:

\begin{verbatim}
c:\Windows\SYSTEM32\ntdll.dll
c:\Windows\SYSTEM32\wow64.dll
c:\Windows\SYSTEM32\wow64cpu.dll
c:\Windows\SYSTEM32\wow64win.dll
c:\Windows\SysWOW64\ntdll.dll
c:\Windows\syswow64\KERNELBASE.dll
c:\Windows\syswow64\kernel32.dll
c:\Windows\syswow64\msvcrt.dll
\end{verbatim}


\subsection{Start from Windows Explorer (file name association)}

If you want to start the computation of transfer functions directly from
the Windows Explorer then you can add a file class definition to the
Windows Registry. It'll extend the context menu entries of the Explorer
for circuit netlist files. With right-click on the file you can load the
netlist file into a text editor and by double-click you can run the
computation with \linnet{}. Please note, if starting \linnet{} this way
Windows will swallow all the program output and you will have to open the
log file after computation to find out if the computation was successful.

Here's a template for a Windows registry script to make an appropriate
file class definition:

% TODO Double-check appropriateness of \clearpage command after any change of chapter Installation
\clearpage
{\footnotesize
\begin{verbatim}
REGEDIT4

[HKEY_CLASSES_ROOT\.cnl]
@="FileClass_cnl"

[HKEY_CLASSES_ROOT\.ckt]
@="FileClass_cnl"

[HKEY_CLASSES_ROOT\FileClass_cnl]
@="linNet Circuit Netlist"

[HKEY_CLASSES_ROOT\FileClass_cnl\shell]
@="Compute"

[HKEY_CLASSES_ROOT\FileClass_cnl\DefaultIcon]
@="c:\\ProgramFiles\\linNet\\linNet.exe,1"

[HKEY_CLASSES_ROOT\FileClass_cnl\shell\Edit]
@="&Edit"
[HKEY_CLASSES_ROOT\FileClass_cnl\shell\Edit\command]
@="c:\\ProgramFiles\\emacs-24.0.92\\bin\\emacsclient.exe -n \"%1\""

[HKEY_CLASSES_ROOT\FileClass_cnl\shell\Compute]
@="Run linNet"
[HKEY_CLASSES_ROOT\FileClass_cnl\shell\Compute\command]
@="c:\\ProgramFiles\\linNet\\linNet.exe -l -c -s -o \"%1\" %*"
\end{verbatim}
} % End of decreased script size for the listing

Copy this code into a text file. Save it using a file name with extension
\file{.reg}, e.g. \file{fileClassCnl.reg}.

Please note, that you will have to modify the proposed code as it uses
absolute paths to the installed applications; replace these paths with
your system's paths. Furthermore, the text editor for editing the circuit
netlists is just a suggestion; you will probably replace the reference to
Emacs with a reference to your favorite text editor.

After customization, double-click this file or run it from a shell window.
If your system has no file name association for \file{*.reg} files then
you need to import it explicitly using a command like \code{regedit.exe
fileClassCnl.reg}. In either case you will need local administrator rights
to complete.