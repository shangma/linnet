\chapter{Building \linnet{}}
\label{secBuildingLinNet}

\linnet{} is distributed as source code together with makefiles to compile
and link the code. The distribution contains ready-to-use binaries for
Windows, for both the win32 and win64 ABI and as debug or production
compilation\footnote{Windows 7 has been used.}. In all other environments
a build of the sources needs to be done prior to the use of \linnet{}.

To start the build you need to \code{cd} in a shell window to the root
directory of the \linnet{} distribution, where the file \file{GNUmakefile}
is located. Here, you issue the command:
\begin{verbatim}
make build
\end{verbatim}
The makefile has some more useful targets and options than just
\ident{build}. Tpye \code{make help} to find out.


\section{Portability of makefiles}

The makefile -- actually it is a set of nested makefiles -- is compatible
with the GNU make processor of at least revision 3.81; 3.80 is not
sufficient. Different derivates of make always tended to be quite
incompatible to one another and other derivates of make will hence need
heavy modifications of the makefiles.

The makefiles directly support the build of the software for Windows with
GCC in the MinGW 32 and 64 Bit ports. The compilation and the compiled
software have been tested with GCC 4.5.2 (32 Bit) and 4.8.1 (64 Bit) and
under Windows 7 only. Other revisions may cause problems: Even between
these two severe incompatibilities have been found, which had to be
tackled by conditional code (see below).

The makefile needs to know, where GCC is installed. For Windows systems
the environment variable \ident{MINGW\_\-HOME} has been introduced. It may
be used to specify the root folder of a MinGW installation. If it is set,
either persistently or -- e.g. to be able to switch between 32 Bit and 64
Bit builds -- on the command line of the make processor then the directory
\file{\$\{MINGW\_\-HOME\}/bin} supersedes the normal system search path. If
the variable is not set then all executables will be located via the
system search path as specified in environment variable PATH. Please refer
to file \file{locateTools.mk} for details.\footnote{Although designed for
Windows, where typically several different competing ports of GCC can
reside, the environment variable can be used for Linux and Mac OS, too.
Don't bother with the name but let the variable point to the parent folder
of GCC's bin folder.}

The makefiles are designed to support the compilation under Linux and Mac
OS also. GNU make's \code{if}/\code{else}/\code{endif} statements have
been applied to do so. This has however never been tested and will
probably not work out of the box. Particularly, you should have a look at
the tool localization in \file{locateTools.mk} and the compiler and linker
command line options in \file{compileAndLink.mk}.


\subsection{Resource compiler}

Under Windows the application icons are added to the executable binary
file. This has no functional aspect; the image data is not accessed from
the functional code. It's just a gimmick that allows to create illustrated
associations between \linnet{} in- and output files and the executable in
the file system browser.

In the makefile the according build rules are placed in
\code{if}/\code{endif} clauses; they should have no effect under either
Linux or Mac OS. Similar mechanisms will probably exist for these operating
systems as well. You may consider to extend the makefiles by adding
appropriate \code{else} paths; an OS indicating variable already exists.


\section{Portability of source code}

The source code itself is system independent; only the abstract functions
from the GNU C library are used.

The compilation with a different compiler tool chain (including another
revision or port of GCC) will probably introduce some changes on the
source files. The applied compiler needs to support the standard C99.


\subsection{GNU extensions}

In module \ident{rat\_rationalNumber} a GNU extension is used: Initialized
structs are used as r-values. This has been done for convenience only, to
achieve at tense and meaningful representation of the required
functionality. It should however be easy and straightforward to eliminate
these (few) constructs. All the rest is according to the standard C99.


\subsection{Incompatibilities of linked libraries}

Far the most necessary code modifications will result from
incompatibilities of the linked libraries. Most obvious is the use of
GCC's Basic Program/System Interface, the command line evaluation support.
All of the command line evaluation related code will probably require a
complete re-implementation with another compiler tool chain. Please refer
to module \ident{opt\_getOpt} for details.

The two tested ports of GCC have different sets of standard functions. The
MinGW 64 Bit port provides \ident{stricmp} while MinGW 32 Bit doesn't. A
simple implementation of this function has been added to the \linnet{}
sources. Depending on the availability of this function in your
environment you will have to modify the according preprocessor switches to
let it or don't let it compile the substitute. Please refer to file
\file{stricmp.h} for details.

The standard function snprintf was found to be buggy in the used 64 Bit
port of GCC. A work around has been implemented, which is compiled only
for the 64 Bit MinGW ports. You will have to check if your revision of GCC
has the same problems and maybe have to modify the preprocessor code that
controls the conditional compilation of the substitute. Please refer to
file \file{snprintf.h} for details.


\section{Re-entrance of code}

The implementation of \linnet{} is widely but not completely reentrant.
This can become an issue if the code should be integrated into a
concurrent environment; for example a client-server structure is set up or
multi-core support is implemented to speed up the computations. As long as
the code is used as of today in a single-threaded self-contained
application the following considerations are irrelevant.

Although the implementation of module \ident{log\_logger} is made in a reentrant
style, the logging process is not reentrant by principle. All logger
instances write to the same global stream \ident{stdout} and concurrent
use would lead to a mess of text at the console. Furthermore, all modules
of \linnet{} use one and the same global logger instance and they hence
share the same file stream, too; here, we'd expect a similar mess.

For module \ident{pci\_parserCircuit}, the parser of the circuit netlist
files, the decision to use a few global variables (e.g. for error
reporting by side effect) has just been taken for convenience, to keep the
code lean. An application for a fully reentrant implementation was simply
not in the scope of the development. It's however straightforward and only
little effort to make the parser reentrant. One just has to put the few
global variables into a struct, which gets the meaning of a parser
instance, add a constructor/destructor pair and pass the pointer to the
parser instance to all the functions.

The situation differs for the other affected module
\ident{rat\_rationalNumber}, the implementation of an exact arithmetics
for rational numbers. Here, the decision to use global data and to loose
re-entrance has been taken intentionally for performance reasons. A global
variable is used to report an overflow. In addition, a global logger
instance is used to visibly report problems. This logger isn't reentrant,
neither. The client code can check the global variable after completing
the operations. The flag is persistent and can thus be checked after a
bunch of operations as an overall result. To avoid the flag we could
return the overflow indication with every operation but this would lead to
a manifold of computational effort. Or we could introduce rational number
processor instances having each their own individual error handlers. The
overhead of passing the instance pointer to all the functions would
probably be less than in the first suggestion but still significant.

If the software is used in an environment, where re-entrance is a must then
serialized access to the module could also be an option. A client is
queued until the other one has completed its (bunch of) operations and has
checked the global error information.