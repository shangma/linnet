\chapter{Formal Syntax Description of Input File}
\label{secBNFOfCnl}

The syntax of the the circuit defining input file (the netlist or
\file{*.cnl} file) in Backus-Naur form:

\begin{small}
\begin{verbatim}
<circuit>       = {[<deviceDef> | <voltageDef> | <resultDef> | <bodeResultDef>] (EOL|';')}
<deviceDef>     = <deviceType> <name> <node> <node> [<node> | <control>] [<relation>]
<deviceType>    = 'R'|'Y'|'C'|'L'|'PI'|'U'[<controlledBy>]|'I'[<controlledBy>]|'OP'
<controlledBy>  = '(' ('U'|'I') ')'
<control>       = <node> <node> | <name>
<relation>      = <name> '=' <quantityRef>
<quantityRef>   = <number> | <deviceRef>
<deviceRef>     = [<rationalNum> '*'] <deviceName>
<voltageDef>    = 'DEF' <name> <node> <node>
<resultDef>     = 'RES' <name> <unknown> {<unknown>} [<plotInfo>]
<bodeResultDef> = 'PLOT' <name> <unknown> <known> [<plotInfo>]
<node>          = <identifier>
<known>         = <name>
<unknown>       = <name>
<name>          = <identifier>
<plotInfo>      = ('LIN' | 'LOG') <integer> <number> <number>
<identifier>    = <character> {<character> | <digit>}
<character>     = 'a' | ... | 'z' | 'A' | ... | 'Z' | '_'
<number>        = <integer> ['.' [<integer>]] [<exponent>]
<rationalNum>   = ['('] <integer> '/' <integer> [')']
<integer>       = <digit> {<digit>}
<exponent>      = ('y' | 'z' | 'a' | 'f' | 'p' | 'n' | 'u' | 'm' | 'c' | 'd'
                   | 'D' | 'h' | 'k' | 'M' | 'G' | 'T' | 'P' | 'X' | 'Z' | 'Y'
                  )
                  | ('E' ['+'|'-'] <integer>)
<digit>         = '0' | ... | '9'
\end{verbatim}
\end{small}

Remarks:

Whitespace separates syntax elements and has not been integrated into
the syntax graph. Blanks and tabs are permitted.

Comments are defined as in C/C++ and are not shown in the syntax graph.

Forward references are generally not supported. This is why result
definitions should appear at the end of the file; they typically refer to
nodes, which need to be already defined. An exception are node references
made in the control inputs of voltage controlled sources; the referenced
nodes may be defined later by being connected to actual devices.

\code{<deviceDef>}: The number of nodes depends:
\begin{itemize}
  \item Most devices have two connectors. They specify two nodes (2 times
    \code{<node>}), which they are connected to
  \item The case having 3 times \code{<node>} is for operational
    amplifiers only
  \item The case having 4 times \code{<node>} is for the voltage
    controlled sources. The first two nodes indicate where the source is
    connected and the second pair (\code{<control>}) defines the
    controlling voltage potential difference
  \item The case having 2 times \code{<node>} followed by a device name
    (\code{<control>}) is for the current controlled sources. The device
    name references an already defined current probe element (type
    \code{PI}). The current through this element is the control current of
    the source
\end{itemize}

\code{<deviceRef>}: \code{<rationalNum>} is a quotient of two positive
integer numbers in the range $[1..999]$. For clarity, the quotient may be
enclosed in parenthesis.

Devices of type \code{U}, \code{I}, \code{OP} and \code{PI} can't have a
value and thus don't have a \code{<relation>}.


Explanation of the specification of the desired computation results:

\code{<voltageDef>}: Definition of any voltage in the network as potential
difference between two nodes. The voltage gets a name.

\code{<resultDef>}: Definition of a full result. Such a result means the
formula how a number of unknowns is composed from all knowns. Named
voltages according to \code{<voltageDef>} and/or any unknown introduced
and named by the LES creator can be referenced. The generated Octave
script code will contain a LTI MIMO transfer function object, which
describes the full set of dependencies between knowns and unknowns.

\code{<bodeResultDef>}: Definition of a frequency response plot. This plot
is limited to the dependency of one quantity on one other one. The result
is implemented as an LTI SISO transfer function object in the generated
Octave script.

\code{<plotInfo>}: The first number is the number of plotted points, in
total for linear frequency axes or per decade for logarithmic axes. The
following pair of numbers is the frequency range.


